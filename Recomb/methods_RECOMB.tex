%!TEX root = main_RECOMB.tex
\section{Methods}
\label{sec:methods}

\newcommand{\PE}[1]{E(#1)}
\newcommand{\EI}{\text{EI}}
\newcommand{\ES}{\text{ES}}
\newcommand{\ISO}{\text{ISO}}

\subsection{Probabilistic model}
Our probabilistic model aims at capturing both the stability of the folded RNA and its ability to adopt a predefined 3D conformation.
To that purpose, a Boltzmann weighted distribution is used, based on a pseudo-energy function $\PE{\cdot}$ which includes contributions for both 
the free-energy and the isostericity. 
More precisely, let $\Omega$ be an gapless RNA alignment, $S$ its associated secondary structure, 
then each sequence $s$ has probability proportional to its Boltzmann factor
\begin{align*}
  \mathcal{B}(s) &= e^\frac{-\PE{s}}{RT} &&\text{with}&\PE{s}&:=\alpha\cdot\ES(s,S)+(1-\alpha)\cdot\EI(s,S,\Omega),
\end{align*}
where $R$ is the Boltzmann constant, $T$ the temperature in Kelvin, $\ES(s)$ and $\EI(s,S,\Omega)$ are the free-energy and isostericity contributiuons, further described below, and $\alpha\in[0,1]$ is an arbitrary parameter that sets the relative weight for both contributions.

\subsubsection{Energy contribution}
The free-energy contribution in our pseudo-energy model corresponds to an additive stacking-pairs model, taking values from the Turner \todo[fancyline]{which version of the Turner model?}XXXXX model retrieved from the NNDB~\cite{Turner2010}. Given a candidate sequence $s$ for a secondary structure $S$, the free-energy of $S$ on $s$ is given by
\begin{align*}
  \ES(s,S) = \sum_{\substack{(i,j)\to (i',j')\in S\\ \text{stacking pairs}}}\ES^{\beta}_{s_is_j\to s_{i'}s_{j'}} 
\end{align*}
where $\ES^{\beta}_{ab\to a'b'}$ is the tabulated free-energy of a stacking pair in the Turner model if available, $0$ if $ab=\varnothing$ (no base-pair to stack onto), and using default value $\beta\in[0,\infty]$ for non-Watson-Crick/Wobble entries (i.e. not in $\{\text{GU},\text{UG},\text{CG},\text{GC}, \text{AU or UA}\}$). This latter parameter allows to choose whether to completely forbid a sequence inducing invalid base pairs ($\beta = \infty$), or to simply penalize it.
The imprecision introduced by this simplification of the Turner model remains reasonable, since the targeted secondary structure is fixed 
(e.g. multiloops do not account for base-specific contributions). Furthermore, it greatly eases the design and implementation of dynamic-programming equations. 
\subsection{isostericity contribution}
The concept of isostericity captures the geometric superimposability of two base-pairs, using individual scores given by Stombaugh~\emph{et al}~\cite{Stombaugh2009} based on a probabilistic model. Let $s$ be a candidate sequence for a secondary structure $S$, given in the context of a gapless RNA alignment $\Omega$,  the contribution of the isostericity to the pseudo-energy is given by
\begin{align*}
  \ES(s,S,\Omega) &= \sum_{\substack{(i,j)\in S\\ \text{pairs}}}\EI^{\Omega}_{(i,j),s_i s_j}, & \text{where}&& 	\EI^{\Omega}_{(i,j),ab}:=
	\frac{
		\sum_{s'\in\Omega}
			\text{\ISO}((s_i',s_j'),(a,b))}
%-		\left(			\text{\ISO}((s_i',s_j'),(s_i,s_j))				\right)	
{		
		|\Omega|
	}
\end{align*}
is the average isostericity of a base-pair in the candidate sequence in comparison with the reference alignment.
The $\ISO$ function uses the \todo[fancyline, inline]{\relsize{-1}Is this true? If so, we might want to try taking the most likely edge/orientation combination for each couple instead.}{Watson-Crick/Watson-Crick} cis isostericity matrix computed by Stombaugh~\emph{et al}~\cite{Stombaugh2009}. Isostericity scores range between $0$ and $9.7$, with $0$ being assigned to a perfect isostericity, and a penalty of $10$ is used for missing entries.
The isostericity contribution will favor exponentially sequences that are likely to adopt a similar local conformation as the alignment sequences.

\subsection{Mutational profiles of sequences}

Let $s$ an RNA 
sequence and $m\geq 0$. $S$ is considered as  one derivation of the \todo[fancyline]{A bit tough: SCFG's were never mentioned before. Maybe introduce this with a sentence relating problem to parsing}{SCFG} generating  
all  secondary structures of length  $|s|$. We are interested in  
the probability of a given position being a specific nucleotide,
over all sequences having at most $m$ mutations from $s$, under the SCFG derivation $S$ 
(formally 
$\mathbb{P}(s_i = x\mid s,\Omega, S,m)$). 
We define as a variant of the
 \emph{Inside-Outside algorithm}~\cite{Lari1990}, allowing us to obtain the
 desired probability,  the two  functions
$\Z{i,j}{m}{a,b}$ and $\Y{i,j}{m}{a,b}$. The former presented in 
Equations~\eqref{eq:Z_in} and~\eqref{eq:Z_rec}, our 
version of the \emph{inside}, computes the partition function between $i$ and $j$ included, 
knowing that position $i-1$ is composed of nucleotide $a$ (resp. $j+1$ is $b$) and containing 
$m$ mutations. The latter in Equations~\eqref{eq:Y_in} and~\eqref{eq:Y_rec},
 computes the \emph{outside}, in particular,  
the partition function considering only between $[0,i]\cup[j,n-1]$
knowing  that position $i+1$ is composed of $a$ (resp. $j-1$ is $b$) and containing 
 $m$ mutations outside.



\subsubsection{Definitions}
Let be $B:=\left\{\text{A},\text{C},\text{G},\text{U}\right\}$, the set of nucleotides.
Given $s\in B^n$ an RNA sequence, let $s_i$ be the nucleotide at position $i$. Let $\Omega$ be a set of un-gapped RNA sequences of
length $n$, and $S$ a secondary structure without pseudoknots. 
Formally, if $(i,j)$ and $(k,l)$ are base pairs in $S$, there is no overlapping extremities
 $\{i,j\}\cap \{k,l\}=\varnothing$ and either the intersection is empty 
 ($[i,j]\cap[k,l]=\varnothing$) or one is included in the other ($[k,l]\subset[i,j]$ or 
 $[i,j]\subset[k,l])$. Let $R$ be the Boltzmann constant, $T$ the temperature in Kelvin and
  the function $\delta$ such that: 
 $\forall a,a' \in B, \delta_{a,a'}:=\left\{\begin{array}{ll}
															1 & \text{If } a\equiv a'\\
															0 & \text{Otherwise.}
														\end{array}\right.$
 Let us finally denote by $E_{(i,j),ab \to ab'}^{\Omega,\beta}$ the local contribution of a base-pair $(i,j)$ to the pseudo-energy, such that
\begin{equation}
  E_{(i,j),ab \to a'b'}^{\Omega,\beta}  = \alpha \cdot\ES^\beta_{a b \to a' b'}+(1-\alpha)\cdot\EI^{\Omega}_{(i,j),a'b'}.
\end{equation}


\subsubsection{Inside}
The \emph{Inside} function $\Z{i,j}{m}{a,b}$ is the partition function considering only the 
energy in subsequence $[i,j]$ over mutants of $s$ having exactly $m$ mutations between $[i,j]$ and whose nucleotide at position $i-1$ is $a$ (resp. $b$ in position $j+1$).
We define function $\Z{i,j}{m}{a,b}$ as a recurrence, and will use as initial conditions:

\begin{equation}
	\forall i \in [0,n-1]:\, \Z{i+1,i}{m}{a,b}=\left\{
	\begin{array}{ll}
		1 &\text{If } m = 0\\
		0 &\text{Otherwise.}
	\end{array}\right.
\label{eq:Z_in}
\end{equation}
In other words, when we evaluate the function $\mathcal Z$, after exhausting all positions, there 
is only one possible solution if there is $0$ mutations left, and none else. Since the 
energetic terms only depend on base pairs, they are not involved in the initial conditions. 

The recursion itself is composed of four terms:
\begin{equation}
	\Z{i,j}{m}{a,b}:=\left\{
  \begin{array}{ll}
  		\displaystyle
      \sum_{\substack{a'\in \B,\\ \Kron_{a',s_i}\le m}}  
      \Z{i+1,j}{m-\Kron_{a',s_i}}{a',b} & \text{If }S_{i}=-1\\
      \displaystyle
      \sum_{\substack{a',b'\in \B^2,\\ \Kron_{a'b',s_is_j}\le m}}
			 e^{\frac{-E_{(i,j),ab \to a'b'}^{\Omega,\beta}}{RT}}
			 \cdot \Z{i+1,j-1}{m-\Kron_{a'b',s_is_j}}{a',b'}&
			 \text{Elif }S_i=j \land S_{i-1}=j+1\\
			 \displaystyle
      \sum_{\substack{a',b'\in \B^2,\\ \Kron_{a'b',s_is_k}\le m}}
      \sum_{m'=0}^{m-\Kron_{a'b',s_is_k}}
   		 e^{\frac{-E_{(i,k),\varnothing\to a'b'}^{\Omega,\beta}}{RT}}
      \cdot\Z{i+1,k-1}{m-\Kron_{a'b',s_is_k}-m'}{a',b'}
      \cdot\Z{k+1,j}{m'}{b',b} & \text{Elif }S_i=k \land i < k \leq j\\
      0 &\text{Otherwise}
	\end{array}\right.
\label{eq:Z_rec}
\end{equation}
The cases can be broken down as follows:
\begin{description}
\item[$S_{i}=-1$:] If the nucleotide at position $i$ is not paired, then the value is the same
as if we increase the lower interval bound by $1$ (i.e. $i+1$), and consider all possible
 nucleotides $a'$ at position $i$, updating the number of mutants in function of $\Kron_{a',s_i}$. 
\item[$S_i=j$ and $S_{i-1}=j+1$:] If nucleotide $i$ is paired with $j$ and nucleotide $i-1$ is
paired with $j+1$, we are in the only case were stacked base pairs can occur. We thus add
the energy of the stacking and of the isostericity of the base pair $(i,j)$. What is left
to compute is the \emph{inside} value of the interval $[i+1,j-1]$ over all possible nucleotides 
$a',b'\in B^2$ at positions $i$ and $j$ respectively.
\item[$S_i=k$ and $i<k \leq j$:] If nucleotide $i$ is paired with position $k$ 
but is not stacked outside, the 
only term contributing directly to the energy is the isostericity of the base pair $(i,k)$. This 
creates
two different intervals for which we must compute the values, $[i+1,k-1]$ and $[k+1,j-1]$, for 
all possible values $a',b'\in B^2$ for nucleotides at positions $i$ and $j$ respectively.
\item[Else:] In any other case, we are in a derivation of the SCFG that does not correspond to the 
secondary structure $S$, and we return $0$.
\end{description}

\subsubsection{Outside}	
The \emph{Outside} function, $\mathcal Y$, is the partition function considering only the 
energy in subsequences $[0,i]\cup[j,n-1]$ over the mutants of $s$ having exactly $m$ mutations between $[0,i]\cup[j,n-1]$ and whose nucleotide at position $i+1$ is $a$ 
(resp. in position $j-1$ it is $b$).
We define function $\Y{i,j}{m}{a,b}$ as a recurrence, and will use as initial conditions:
\begin{equation}
	\Y{-1,j}{m}{X,X}:=
		\displaystyle
	  \Z{j,n-1}{m}{X,X}
\label{eq:Y_in}
\end{equation}
The recurrence, as shown below, will increase the interval $[i,j]$ by decreasing $i$ when
it is not base paired. If it is with a position $k>j$, we increase $j$ to include it.
 Thus, when we need
to evaluate an interval as $(-1,j)$, all stems between $(0,j)$ are taken into account and the
structure between $(j,n-1)$ must be a set of independent stems. Therefore,
 all the outside energy between $[j,n-1]$ is
equal to $\Z{j,n-1}{m}{X,X}$, for any $X\in B$. The recursion itself is as follows.
\begin{equation}
	\Y{i,j}{m}{a,b} = \left\{
  \begin{array}{ll}
		\displaystyle
    \sum_{\substack{a'\in \B,\\ \Kron_{a',s_i}\le m}}
    \Y{i-1,j}{m- \Kron_{a',s_i}}{a',b} &
    \text{Elif }S_i=-1 \\
    \displaystyle
    \sum_{\substack{a'b'\in \B^2,\\ \Kron_{a'b',s_is_j}\le m}}
		 e^{\frac{-E_{(i,j),ab \to a'b'}^{\Omega,\beta}}{RT}}\cdot
    \Y{i-1,j+1}{m- \Kron_{a'b',s_is_j}}{a',b'} &
   	 \text{Elif }S_{i}=j \land S_{i+1}=j-1\\
		 \displaystyle
		 \sum_{\substack{a'b'\in \B^2,\\ \Kron_{a'b',s_is_k}\le m}}
		 \sum_{m'=0}^{m-\Kron_{a'b',s_is_k}}
  		 e^{\frac{-E_{(i,k),\varnothing\to a'b'}^{\Omega,\beta}}{RT}}
		 \cdot\Y{i-1,k+1}{m- \Kron_{a'b',s_is_k} - m'}{a',b'}
     \cdot\Z{j,k-1}{m'}{b,b'} &
		 \text{Elif }S_{i}=k \geq j\\
		 \displaystyle
		 \sum_{\substack{a'b'\in \B^2,\\ \Kron_{a'b',s_ks_i}\le m}}
		 \sum_{m'=0}^{m-\Kron_{a'b',s_ks_i}}
   	 e^{\frac{-E_{(k,i),\varnothing\to a'b'}^{\Omega,\beta}}{RT}}
		 \cdot\Y{k-1,j}{m- \Kron_{a'b',s_ks_i} - m'}{a',b}
     \cdot\Z{k+1,i-1}{m'}{a',b'} &
		 \text{Elif }-1 < S_{i}=k < i\\
		 0 & \text{Otherwise}
  \end{array}\right.
\label{eq:Y_rec}
\end{equation}
The five cases can be broked down as follows.
\begin{description}
\item[$S_i=-1$:] If the nucleotide at position $i$ is not paired, then the value is the same
as if we decrease the lower interval bound by $1$ (i.e. $i-1$), and consider all possible
nucleotides $a'$ at position $i$, correcting the number of mutants
in function of $\Kron_{a',s_i}$.
\item[$S_{i}=j$ and $S_{i+1}=j-1$:] If nucleotide $i$ is paired with $j$ and nucleotide $i+1$ is
paired with $j-11$, we are in the only case were stacked base pairs can occur. We thus add
the energy of the stacking and of the isostericity of the base pair $(i,j)$. What is left
to compute is the \emph{outside} value for the interval $[i-1,j+1]$ over all possible nucleotides 
$a',b'\in B^2$ at positions $i$ and $j$ respectively.
\item[$S_{i}=k \geq j$:]If nucleotide $i$ is paired with position $k\geq j$, 
and is not stacked inside, the 
only term contributing directly to the energy is the isostericity of the base pair $(i,k)$. 
We can then consider the outside interval $[i-1,k+1]$ by multiplying it by the the \emph{forward}
value of the newly included interval (i.e. $[j,k-1]$), for 
all possible values $a',b'\in B^2$ for nucleotides at positions $i$ and $k$ respectively.
\item[$-1<S_{i}<i$:]As above but if position $i$ is to pairing with a lower value.
\item[Else:] In all other cases, we are in a derivation of the SCFG that does not correspond to the 
secondary structure $S$, and we return $0$.


\end{description}

\subsubsection{Inside-Outside}
By construction, the partition function over all sequences at exactly $m$ mutations of $s$ can 
be described in function of the \emph{forward} term as $\Z{0,n-1}{m}{X,X}$,
 for any nucleotide $X\in B$ or
in function of the \emph{backward} term, for any unpaired position $k$:
$$
	\Z{0,n-1}{m}{X,X}
	\equiv
	\sum_{\substack{a\in \B,\\ \Kron_{a,s[k]}\le m}}	
	\Y{k-1,k+1}{m-\Kron_{a,s[k]}}{a,a}
$$

We are now left to compute the probability that a \todo[fancyline]{Maybe worth doing it for base-pairs? maybe not\ldots}{given position is a given nucleotide}.
We leverage the \emph{Inside-Outside} construction to immediately obtain the following $3$ cases.
Given $i\in[0,n-1],x\in B$, and $M\geq 0$ a bound on the number of allowed mutations. 
\begin{align*}
	\mathbb{P}(s_i = x\mid s,\Omega, S,M) &:= \frac{\mathcal{B}(i,x,s,\Omega,S,M)}{\sum_{m=0}^{M}\Z{0,n-1}{m}{X,X}}\\ 
\mathcal{B}(i,x,s,\Omega,S,M)&=
 \left\{
	\begin{array}{ll}
			\sum_{m=0}^{M}
			\Y{i-1,i+1}{m-\Kron_{x,s_i}}{x,x}
		&\text{If }S_i = -1\\
			\sum_{m=0}^{M}
			\sum_{\substack{b\in B\\\Kron_{xb,s_is_k\leq m}}}
			\sum_{m'=0}^{m-\Kron_{xb,s_is_k}}
     	 e^{\frac{-E_{(i,k),\varnothing\to xb}^{\Omega,\beta} }{RT}}
			\Y{i-1,k+1}{m-\Kron_{xb,s_is_k-m'}}{x,b}
			\Z{i+1,k-1}{m'}{x,b}
		&\text{If }S_i=k>i\\
			\sum_{m=0}^{M}
			\sum_{\substack{b\in B\\\Kron_{bx,s_ks_i\leq m}}}
			\sum_{m'=0}^{m-\Kron_{bx,s_ks_i}}
     	 e^{-E_{(k,i),\varnothing\to bx}{RT}}
			\Y{k-1,i+1}{m-\Kron_{bx,s_ks_i-m'}}{b,x}
			\Z{k+1,i-1}{m'}{b,x}
		&\text{If }S_i=k<i
	\end{array}\right.
\end{align*}

In every case, the denominator is the sum of the partitions function of exactly $m$ mutations, \todo[fancyline,inline]{\relsize{-1}Seems weird to do this for a given upper bound $M$ (as opposed to exact $m$), and it makes the equations a bit more complicated. Anyway, there is roughly $3n$ times more sequences with $m+1$ mutations than with $m$ muts, so I expect the $m+1$ guys are so dominating in the final probabilities that it may not be worth the trouble of combining them with the \emph{little guys} ($\#muts < m$)}{for $m$ smaller or equal to our target $M$}. The numerators are divided in the following three cases.
\begin{description}
\item[$S_i=-1$:] If the nucleotide at position $i$ is not paired, we are concerned by the weights
over all sequences which have at position $i$ nucleotide $x$, which is exactly the sum
of the values of $\Y{i-1,i+1}{m-\Kron_{x,s_i}}{x,x}$, for all $m$ between $0$ and $M$.
\item[$S_i=k>i$:] Since we need to respect the derivation of the secondary structure $S$, if 
position $i$ is paired, we must consider the two partition functions. The \emph{outside} of the 
base pair, and the \emph{inside}, for all possible values for the nucleotide at position $k$, and
all possible distribution of the mutant positions between the inside and outside of the base pair. We also add the term of isostericity for this specific base pair.
\item[$S_i=k<i$:] Same as above, but with position $i$ pairing with a lower position.
\end{description}