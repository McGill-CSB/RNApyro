%!TEX root = main_RECOMB.tex
\section{Methods}
\label{sec:methods}
Let $\Omega$ be an un-gapped RNA alignment, $S$ its associated secondary structure, $s$ an RNA 
sequence and $m\geq 0$. $S$ is considered as  one derivation of the SCFG generating  
all possible secondary structures of length  $s$. We are interested in  
the probability of a given position being a specific nucleotides,
over all sequences having at most $m$ mutations from $s$, under the SCFG derivation $S$ (i.e.
$\mathbb{P}(s_i = x\mid s,\Omega, S,m)$). 
We define a variant of the
 \emph{Inside-Outside algorithm}~\cite{Lari1990}, allowing us to obtain the
 desired probability, with the two following functions,
$\Z{i,j}{m}{a,b}$ and $\Y{i,j}{m}{a,b}$. The former, as the 
equivalent of the \emph{inside}, computes the partition function between $i$ and $j$ included, 
knowing that position $i-1$ is composed of nucleotide $a$ (resp. $j+1$ is $b$) and containing $m$ mutations. The latter computes the \emph{outside}, in particular, knowing 
that position $i+1$ is composed of $a$ (resp. $j-1$ is $b$) and containing outside,
including $i$ and $j$, $m$ mutations.
The Boltzmann weights are a combination of the base pairs stacking energy, 
using as values those of the NNDB~\cite{Turner2010}, and
the average isostericity with $\Omega$, as defined in~\cite{Stombaugh2009}. 


\subsection{Definitions}
Let be $B:=\left\{\text{A},\text{C},\text{G},\text{U}\right\}$, the set of nucleotides.
Given $s\in B^n$ an RNA sequence, let $s_i$ be the nucleotide at position $i$. Let $\Omega$ be a set of un-gapped RNA sequences of
length $n$, and $S$ a secondary structure without pseudoknots. 
Formally, if $(i,j)$ and $(k,l)$ are base pairs in $S$, there is no overlapping extremities
 $\{i,j\}\cap \{k,l\}=\varnothing$ and either the intersection is empty 
 ($[i,j]\cap[k,l]=\varnothing$) or one is included in the other ($[k,l]\subset[i,j]$ or 
 $[i,j]\subset[k,l])$. Let $R$ be the Boltzmann constant, $T$ the temperature in Kelvin and
  the function $\delta$ such that: 
 $\forall a,a' \in B, \delta_{a,a'}:=\left\{\begin{array}{ll}
															1 & \text{If } a\equiv a'\\
															0 & \text{Else}
														\end{array}\right.$

\subsection{Energy Model}
The energy we will be composed of two function, $\text{ES}^{\beta}_{ab\to a'b'}$ and 
$\text{EI}^{\Omega}_{(i,j),ab}$. The former is equal to the 
stacking energy of the base pair with nucleotides $ab$ on top of the base pair with nucleotides 
$a'b'$, as set in the NNDB~\cite{Turner2010}. If one of the base pair is not valid (i.e. not in 
$\{\text{GU},\text{UG},\text{CG},\text{GC}, \text{AU or UA}\}$, the value is a parameter 
$\beta \in [1,\infty]$. This allows
to completely forbid a sequence where a base pair is non valid, when $\beta = \infty$ or only 
penalize it.
$\text{EI}^{\Omega}_{(i,j), a'b'}$ is the average of the sum of differences between the isostericity
of base pairs at positions $(i,j)$ in $\Omega$ and $s_is_j$, and the isostericity of base pairs
 at positions $(i,j)$ in $\Omega$ and $ab$. If gives us an indication 
 if the base pair $ab$ is more isosteric to the set $\Omega$ than the one on the sequence 
 $\omega$. Formally, given $s$, $\Omega$, two positions $(i,S_i)$ and two nucleotides $a,b\in B$:
 $$
 	\text{EI}^{\Omega}_{(i,j),ab}:=
	\frac{
		\displaystyle
		\sum_{s'\in\Omega}
		\left(
			\text{ISO}((s_i',s_j'),(a,b))
		\right)-
		\left(
			\text{ISO}((s_i',s_j'),(s_i,s_j))		
		\right)
	}{
		\displaystyle
		|\Omega|
	}
 $$
  Where $\text{ISO}((a',b'),(a,b))$ is the isostericity value 
 between  the canonical base pairs $(a',b')$ and $(a,b)$  as defined in~\cite{Stombaugh2009}. 
 Let be $\alpha\in[0,1]$, it 
 will be used to balance the weight given to the stacking energy and the isostericity.
	
\subsection{Inside}
The \emph{Inside} function $\Z{i,j}{m}{a,b}$ is the partition function considering only the 
energy in subsequence $[i,j]$ over mutants of $s$ having exactly $m$ mutations between $[i,j]$ and whose nucleotide at position $i-1$ is $a$ (resp. in position $j+1$ it is $b$).
We define function $\Z{i,j}{m}{a,b}$ as a recurrence, and will use as initial conditions:

\[
	\forall i \in (0,\cdots,n-1):\, \Z{i+1,i}{m}{a,b}=\left\{
	\begin{array}{ll}
		1 &\text{If } m = 0\\
		0 &\text{Else }
	\end{array}\right.
\]
In other words, when we evaluate the function $\mathcal Z$, after exhausting all positions, there 
is only one possible solution if there is $0$ mutations left, and none else. Since the 
energetic terms only depend on base pairs, they are not involved in the initial conditions. 

The recursion itself is composed of four terms:
$$
	\Z{i,j}{m}{a,b}:=\left\{
  \begin{array}{ll}
  		\displaystyle
      \sum_{\substack{a'\in \B,\\ \Kron_{a',s_i}\le m}}  
      \Z{i+1,j}{m-\Kron_{a',s_i}}{a',b} & \text{If }S_{i}=-1\\
      \displaystyle
      \sum_{\substack{a',b'\in \B^2,\\ \Kron_{a'b',s_is_j}\le m}}
			 e^{\frac{-(\alpha \text{ES}^\beta_{a b \to a' b'}+(1-\alpha)\text{EI}^{\Omega}_{(i,j),a'b'})}{RT}}
			 \Z{i+1,j-1}{m-\Kron_{a'b',s_is_j}}{a',b'}&
			 \text{Elif }S_i=j \land S_{i-1}=j+1\\
			 \displaystyle
      \sum_{\substack{a',b'\in \B^2,\\ \Kron_{a'b',s_is_k}\le m}}
      \sum_{m'=0}^{m-\Kron_{a'b',s_is_k}}
   		 e^{\frac{-(1-\alpha)\text{EI}^{\Omega}_{(i,k),a'b'}}{RT}}
      \Z{i+1,k-1}{m-\Kron_{a'b',s_is_k}-m'}{a',b'}
      \Z{k+1,j}{m'}{b',b} & \text{Elif }S_i=k \land i < k \leq j\\
      0 &\text{Else}
	\end{array}\right.
$$
The cases can be broked down as follows.
\begin{description}
\item[$S_{i}=-1$:] If the nucleotide at position $i$ is not paired, then the value is the same
as if we increase the lower interval bound by $1$ (i.e. $i+1$), and consider all possible
 nucleotides $a'$ at position $i$. 
\item[$S_i=j$ and $S_{i-1}=j+1$:] If nucleotide $i$ is paired with $j$ and nucleotide $i-1$ is
paired with $j+1$, we are in the only case were stacked base pairs can occur. We thus add
the energy of the stacking and of the isostericity of the base pair $(i,j)$. What is left
to compute is the \emph{inside} value of the interval $[i+1,j-1]$ over all possible nucleotides 
$a',b'\in B^2$ at positions $i$ and $j$ respectively.
\item[$S_i=k$ and $i<k \leq j$:] If nucleotide $i$ is paired with position $k$ 
but is not stacked outside, the 
only term contributing directly to the energy is the isostericity of the base pair $(i,k)$. This 
creates
two different intervals for which we must compute the values, $[i+1,k-1]$ and $[k+1,j-1]$, for 
all possible values $a',b'\in B^2$ for nucleotides at positions $i$ and $j$ respectively.
\item[Else:] In all other cases, we are in a derivation of the SCFG that does not correspond to the 
secondary structure $S$, and we return $0$.
\end{description}

\subsection{Outside}	
The \emph{Outside} function, $\mathcal Y$, is the partition function considering only the 
energy in subsequences $[0,i]\cup[j,n-1]$ over the mutants of $s$ having exactly $m$ mutations between $[0,i]\cup[j,n-1]$ and whose nucleotide at position $i+1$ is $a$ 
(resp. in position $j-1$ it is $b$).
We define function $\Y{i,j}{m}{a,b}$ as a recurrence, and will use as initial conditions:
$$
	\Y{-1,j}{m}{X,X}:=
		\displaystyle
	  \Z{j,n-1}{m}{X,X}
$$
The recurrence, as shown below, will increase the interval $[i,j]$ by decreasing $i$ when
it is not base paired. If it is with a position $k>j$, we increase $j$ to include it.
 Thus, when we need
to evaluate an interval as $(-1,j)$, all stems between $(0,j)$ are taken into account and the
structure between $(j,n-1)$ must be a set of independent stems. Thus, all the outside energy is
equal to $\Z{j,n-1}{m}{X,X}$, for any $X\in B$. The recursion itself is the following:
$$
	\Y{i,j}{m}{a,b} = \left\{
  \begin{array}{ll}
		\displaystyle
    \sum_{\substack{a'\in \B,\\ \Kron_{a',s_i}\le m}}
    \Y{i-1,j}{m- \Kron_{a',s_i}}{a',b} &
    \text{Elif }S_i=-1 \\
    \displaystyle
    \sum_{\substack{a'b'\in \B^2,\\ \Kron_{a'b',s_is_j}\le m}}
		 e^{\frac{-(\alpha \text{ES}^\beta_{a b \to a' b'}+(1-\alpha)\text{EI}^{\Omega}_{(i,j),a'b'})}{RT}}
    \Y{i-1,j+1}{m- \Kron_{a'b',s_is_j}}{a',b'} &
   	 \text{Elif }S_{i}=j \land S_{i+1}=j-1\\
		 \displaystyle
		 \sum_{\substack{a'b'\in \B^2,\\ \Kron_{a'b',s_is_k}\le m}}
		 \sum_{m'=0}^{m-\Kron_{a'b',s_is_k}}
  		 e^{\frac{-(1-\alpha)\text{EI}^{\Omega}_{(i,k),a'b'}}{RT}}
		 \Y{i-1,k+1}{m- \Kron_{a'b',s_is_k} - m'}{a',b'}
     \Z{j,k-1}{m'}{b,b'} &
		 \text{Elif }S_{i}=k \geq j\\
		 \displaystyle
		 \sum_{\substack{a'b'\in \B^2,\\ \Kron_{a'b',s_ks_i}\le m}}
		 \sum_{m'=0}^{m-\Kron_{a'b',s_ks_i}}
   	 e^{\frac{-(1-\alpha)\text{EI}^{\Omega}_{(k,i),a'b'}}{RT}}
		 \Y{k-1,j}{m- \Kron_{a'b',s_ks_i} - m'}{a',b}
     \Z{k+1,i-1}{m'}{a',b'} &
		 \text{Elif }-1 < S_{i}=k < i\\
		 0 & \text{Else}
  \end{array}\right.
$$
The five cases can be broked down as follows.
\begin{description}
\item[$S_i=-1$:] If the nucleotide at position $i$ is not paired, then the value is the same
as if we decrease the lower interval bound by $1$ (i.e. $i-1$), and consider all possible
nucleotides $a'$ at position $i$.
\item[$S_{i}=j$ and $S_{i+1}=j-1$:] If nucleotide $i$ is paired with $j$ and nucleotide $i+1$ is
paired with $j-11$, we are in the only case were stacked base pairs can occur. We thus add
the energy of the stacking and of the isostericity of the base pair $(i,j)$. What is left
to compute is the \emph{outside} value for the interval $[i-1,j+1]$ over all possible nucleotides 
$a',b'\in B^2$ at positions $i$ and $j$ respectively.
\item[$S_{i}=k \geq j$:]If nucleotide $i$ is paired with position $k\geq j$, 
and is not stacked inside, the 
only term contributing directly to the energy is the isostericity of the base pair $(i,k)$. 
We can then consider the outside interval $[i-1,k+1]$ by multiplying it by the the \emph{forward}
value of the newly included interval (i.e. $[j,k-1]$), for 
all possible values $a',b'\in B^2$ for nucleotides at positions $i$ and $k$ respectively.
\item[$-1<S_{i}<i$:]As above but if the pairing is to the left.
\item[Else:] In all other cases, we are in a derivation of the SCFG that does not correspond to the 
secondary structure $S$, and we return $0$.


\end{description}

\section{Inside-Outside}
By construction, the partition function over all sequences at exactly $m$ mutations of $s$ can 
be described in function of the \emph{forward} term as $\Z{0,n-1}{m}{X,X}$,
 for any nucleotide $X\in B$ or
in function of the \emph{backward} term, for any position $k$ such that $S_k=-1$:
$$
	\Z{0,n-1}{m}{X,X}
	\equiv
	\sum_{\substack{a\in \B,\\ \Kron_{a,s[k]}\le m}}	
	\Y{k-1,k+1}{m-\Kron_{a,s[k]}}{a,a}
$$

We are now interested in knowing, under our model, 
 the probability that a given position is a given nucleotide.
We leverage the \emph{Inside-Outside} construction to immediately obtain the following $3$ cases.
Given $i\in[0,n-1],x\in B$, and $M\geq 0$ a bound on the number of mutations allowed. 
$$
	\mathbb{P}(s_i = x\mid s,\Omega, S,M):=\left\{
	\begin{array}{ll}
		\displaystyle
		\frac{
			\displaystyle
			\sum_{m=0}^{M}
			\Y{i-1,i+1}{m-\Kron_{x,s_i}}{x,x}
		}{
			\displaystyle
			\sum_{m=0}^{M}
			\Z{0,n-1}{m}{X,X}
		}
		&\text{If }S_i = -1\\
		\displaystyle
		\frac{
			\displaystyle
			\sum_{m=0}^{M}
			\sum_{\substack{b\in Bases\\\Kron_{xb,s_is_k\leq m}}}
			\sum_{m'=0}^{m-\Kron_{xb,s_is_k}}
     	 e^{\frac{-(1-\alpha)\text{EI}^{\Omega}_{(i,k),xb}}{RT}}
			\Y{i-1,k+1}{m-\Kron_{xb,s_is_k-m'}}{x,b}
			\Z{i+1,k-1}{m'}{x,b}
		}{
			\displaystyle
			\sum_{m=0}^{M}
			\Z{0,n-1}{m}{X,X}
		}
		&\text{If }S_i=k>i\\
		\displaystyle
 		\frac{
			\displaystyle
			\sum_{m=0}^{M}
			\sum_{\substack{b\in Bases\\\Kron_{bx,s_ks_i\leq m}}}
			\sum_{m'=0}^{m-\Kron_{bx,s_ks_i}}
     	 e^{\frac{-(1-\alpha)\text{EI}^{\Omega}_{(k,i),bx}}{RT}}
			\Y{k-1,i+1}{m-\Kron_{bx,s_ks_i-m'}}{b,x}
			\Z{k+1,i-1}{m'}{b,x}
		}{
			\displaystyle
			\sum_{m=0}^{M}
			\Z{0,n-1}{m}{X,X}
		}
		&\text{If }S_i=k<i
	\end{array}\right.
$$

In every case, the denominator is the sum of the partitions function of exactly $m$ mutations, for $m$ smaller or equal to our target $M$. The numerators are divided in the following three cases.
\begin{description}
\item[$S_i=-1$:] If the nucleotide at position $i$ is not paired, we are concerned by the weights
over all sequences which have at position $i$ nucleotide $x$, which is exactly the sum
of the values of $\Y{i-1,i+1}{m-\Kron_{x,s_i}}{x,x}$, for all $m$ between $0$ and $M$.
\item[$S_i=k>i$:] Since we need to respect the derivation of the secondary structure $S$, if 
position $i$ is paired, we must consider the two partition functions. The \emph{outside} of the 
base pair, and the \emph{inside}, for all possible values for the nucleotide at position $k$, and
all possible distribution of the mutant positions between the inside and outside of the base pair. We also add the term of isostericity for this specific base pair.
\item[$S_i=k<i$:] Same as above, but with $i$ pairing with a lower position.
\end{description}