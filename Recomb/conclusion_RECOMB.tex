%!TEX root = main_RECOMB.tex
\section{Conclusion}
\label{sec:conclusion}

In this article we presented a new and efficient way of
exploring the mutational landscape of an RNA, benefit from the
 conserved consensus secondary structures  information,
to identify and fix sequencing errors. We also develop a new
 pseudo energy model, taking into account the nearest-neighbour energy model 
and the isostericity,  quantifying geometrical differences, of all base pairs. 

By combining those two approaches,  the 
mutational landscape exploration and the pseudo energy model,
 we are able  to efficiently 
correct randomly mutated 5s rRNA sequences. 
As presented in Sec.~\ref{sec:results},
we observe that the models
with larger weights on the
isostericity seems to hold a higher accuracy on the estimation of errors.
Importantly, the implementation is fast enough for practical applications.

We must point out that our approach can only detect point wise 
sequencing errors, and only if they are located in base pairs. It 
nonetheless supplements well existing tools by using previously discarded
information.

