%!TEX root = main_RECOMB.tex
\section{Conclusion}
\label{sec:conclusion}

In this article we presented a new and efficient way of
exploring the mutational landscape of an RNA, benefiting from the
 conserved consensus secondary structures  information,
to identify and fix sequencing errors. We additionally introduce a new
 pseudo energy model, both taking into account the nearest-neighbour energy model 
and, to account for geometrical discrepancies,  the isostericity of all base pairs. 
The algorithm runs in  $\Theta(n\cdot(|\Omega|+m^2))$ time and $\Theta(n\cdot(|\Omega|+m))$ memory, where $n$ is the length of the RNA,
$m$ the number of mutations and $\Omega$ the size of the multiple sequence alignment.


By combining  into \texttt{RNApyro} these two approaches,  the 
mutational landscape exploration and the pseudo energy model,
 we created a tool predicting the positions
 yielding point-wise sequencing error and correcting them.
We validated our model with the 5s rRNA,
as presented in Sec.~\ref{sec:results}.
We observed that the models
with larger weights on the
isostericity seems to hold a higher accuracy on the estimation of errors.
This indicates that an extractable signal is contained in the isostericity.
Importantly, the implementation is fast enough for practical applications. 


We must recall that our approach is restricted to
 the correction of point-wise error in structured regions (i.e. base paired nucleotides).
 Nonetheless it should supplement well existing tools, by using previously discarded
information holding, as shown, a strong signal.

Further research, given the potential of error-correction of \texttt{RNApyro}, 
will evaluate its impact over large datasets with different existing
  NGS error-correction pipe-line.