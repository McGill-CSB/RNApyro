%!TEX root = main_RECOMB.tex
\section{Conclusion}
\label{sec:conclusion}

In this article we presented a new and efficient way of
exploring the mutational landscape of an RNA, benefiting from the
 conserved consensus secondary structures  information,
to identify and fix sequencing errors. We introduced a new
 pseudo energy model, both taking into account the nearest-neighbour energy model 
and, to account for geometrical discrepancies,  the isostericity of all base pairs. 

By combining these two approaches,  the 
mutational landscape exploration and the pseudo energy model,
 into \texttt{RNApyro}, we obtained a tool to predict the position and 
correct randomly mutated 5s rRNA sequences. 
As presented in Sec.~\ref{sec:results},
we observe that the models
with larger weights on the
isostericity seems to hold a higher accuracy on the estimation of errors.
Importantly, the implementation is fast enough for practical applications.

We must recall that our approach is restricted to
 the correction of point-wise error in structured regions (i.e. base paired nucleotides).
 Nonetheless it should supplement well existing tools, by using previously discarded
information holding, as shown, a strong signal.

