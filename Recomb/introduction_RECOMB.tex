%!TEX root = main_RECOMB.tex
\section{Introduction}
\label{sec:introduction}

Ribonucleic acids (RNAs) are  now  an ubiquitous class of molecules, being
found in every living organisms and having a broad range of functions, from catalyzing
chemical reactions as the RNase P or the group II introns,
hybridizing  messenger RNA to regulate gene expression,
to ribosomal RNA (rRNA) synthesizing proteins.
Those functions  require specific structures, 
encoded in their nucleotide sequence. Although the functions
need to be preserved through various organisms, and therefore
their structure must be similar,  the sequences
can greatly differ from one organism to another.
For half a century, biological molecules have been studied as a proxy to understand
evolution~\cite{Zuckerkandl1965}, and with all their characteristics, rRNAs have
become a prime candidate for phylogenetic studies~\cite{Olsen1986, Olsen1993}.

In recent years, studies as the \emph{Human Microbiome Project}~\cite{Turnbaugh2007}, 
leveraging the NGS techniques to sequence as many new organisms 
as possible, are producing a wealth of new information. Although
those techniques have a huge throughput, they yield a sequencing error rate of around
$4\%$~\cite{Huse2007}. This error can be highly reduced in genome projects when highly 
redundant consensus 
assemblies are available, but in studies of new or not well known organisms, there is not
 enough  similarity to differentiate between the sequencing errors and the natural 
 polymorphisms that we want to observe, often inflating the diversity estimates~\cite{Kunin2010}.
  In rRNAs, we have as additional 
 information the conserved secondary structure, and we want to use it to identify
highly probable sequencing errors.
 
The first challenge is to efficiently explore the mutant space, which grows exponentially. 
Leveraging the techniques  in \texttt{RNAmutants}~\cite{Waldispuhl2008}, we develop 
here 