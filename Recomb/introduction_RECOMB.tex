%!TEX root = main_RECOMB.tex
\section{Introduction}
\label{sec:introduction}

Ribonucleic acids (RNAs) are found in every living organism, and exhibit a broad range of functions, ranging from catalyzing
chemical reactions, as the RNase P or the group II introns, hybridizing  messenger RNA to regulate gene expressions,
to ribosomal RNA (rRNA) synthesizing proteins. Those functions  require specific structures,  encoded in their nucleotide
sequence. Although the functions, and thus the structures, need to be preserved through various organisms, the sequences
can greatly differ from one organism to another. This sequence diversity coupled with the structural conservation is a fundamental
asset for evolutionary studies. To this end, algorithms to analyze the relationship between RNA mutants and structures are required.

For half a century, biological molecules have been studied as a proxy to understand evolution~\cite{Zuckerkandl1965}, and due
to their fundamental functions and remarkably conserved structures, rRNAs have always been a prime candidate for phylogenetic
studies~\cite{Olsen1986, Olsen1993}. In recent years, studies as the \emph{Human Microbiome Project}~\cite{Turnbaugh2007} benefited
of new technologies such as the NGS techniques to sequence as many new organisms as possible and extract an unprecedented flow of new information. 
Nonetheless, these high-throughput techniques typically have high error rates that make their applications to metagenomics (a.k.a. environmental
genomics) studies challenging. For instance, pyrosequencing as implemented by Roche's 454 produces may have an error rate raising up to 10\%.
Because there is no cloning step, resequencing to increase accuracy is not possible and it is therefore vital to disentangle noise from true sequence
diversity in this type of data \cite{Quince:2009uq}. Errors can be significantly reduced  when large multiple sequence alignments with close homologs
are available, but in studies of new or not well known organisms, such information is rather sparse. In particular, it is common that there is not enough  similarity to 
differentiate between the sequencing errors and the natural polymorphisms that we want to observe, often leading to artificially inflated diversity estimates~\cite{Kunin2010}.
A few techniques have been developed to remedy to this problem~\cite{Quinlan2008,Medvedev2011} but they do not take
into account all the available information. It is therefore essential to develop methods that can exploit any type of signal available to correct errors.  

In this paper, we introduce \RNApyro, a novel algorithm that enables us to calculate precisely mutational probabilities in RNA sequences with a
conserved consensus secondary structure. We show how our techniques can exploit the structural information embedded in physics-based energy models, 
covariance models and isostericity scales to identify and correct point-wise errors in RNA molecules with conserved secondary structure. In particular, we 
hypothesize that  conserved consensus secondary structures combined with sequence profiles  provide an information that allow us to identify and fix sequencing errors.

Here, we expand the range of algorithmic techniques previously introduced with \RNAmutants~\cite{Waldispuhl2008}.
Instead of exploring the full conformational landscape and sample mutants, we develop an inside-outside algorithm that enables us
to explore the complete mutational landscape with a \emph{fixed} secondary structure and to calculate exactly mutational probability values. In addition
to a gain into the numerical precision, this strategy allows us to drastically reduce the computational complexity ($\mathcal{O}(n^3 \cdot m^2)$ for the
original version of  \RNAmutants to $\mathcal{O}(n \cdot m^2)$ for \RNApyro, where $n$ is the size of the sequence and $m$ the number of mutations).

We design a new scoring scheme combining nearest-neighbor models \cite{Turner2010} to isostericity metrics \cite{Stombaugh2009}.
Classical approaches use a Boltzmann distribution whose weights are estimated using a nearest-neighbour energy model~\cite{Turner2010}. However, the
latter only accounts for  canonical and wobble, base pairs. As was shown by Leontis and Westhof~\cite{Leontis2001},
the diversity of base pairs observed in tertiary structures is much larger, albeit their energetic contribution remains unknown. To quantify geometrical discrepancies, 
an isostericity distance has been designed \cite{Stombaugh2009}, increasing as two base pairs geometrically differ from each other in space. Therefore, we 
incorporate these scores in the Boltzmann weights used by \RNApyro.
 
We illustrate and benchmark our techniques for point-wise error corrections on the 5S ribosomal RNA. We choose the latter since it has been extensively
used for phylogenetic reconstructions~\cite{Hori1987} and its sequence has been recovered for over 712 species (in the Rfam seed alignment with id
\texttt{RF00001}). Using a leave one out strategy, we perform random distributed mutations on a sequence. While our methodology is restricted to the correction of 
point-wise error in structured regions (i.e. with base pairs), we show that \texttt{RNApyro} can successfully extract a signal that can be used to reconstruct the 
original sequence with an excellent accuracy. This suggests that \RNApyro is a promising algorithm to complement existing tools in the NGS error-correction 
pipeline.

The algorithm and the scoring scheme are presented in Sec.~\ref{sec:methods}. Details of the implementation and benchmarks are in Sec.~\ref{sec:results}. 
Finally, we discuss future developments and applications in Sec.~\ref{sec:conclusion}.