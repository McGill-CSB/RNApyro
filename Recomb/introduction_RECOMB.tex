%!TEX root = main_RECOMB.tex
\section{Introduction}
\label{sec:introduction}

A particular class of molecules,
ribonucleic acids (RNAs) are  now  an ubiquitous class of molecules, being
found in every living organisms and having a broad range of functions, from catalyzing
chemical reactions as the RNase P or the group II introns,
hybridizing  messenger RNA to regulate gene expression,
to ribosomal RNA (rRNA) synthesizing protein.
Those functions  require specific structures, 
encoded in their nucleotide sequence. Although the functions
need to be preserved through various organisms, and therefore
their structure must be similar,  the sequences
can greatly differ from one another.
For half a century, biological molecules have been studied as a proxy to understand
evolution~\cite{Zuckerkandl1965}, and with all their characteristics, rRNAs have
become a prime candidate to study phylogeny~\cite{Olsen1986, Olsen1993}.

In recent years, studies as the \emph{Human Microbiome Project}~\cite{Turnbaugh2007}, 
leveraging the NGS techniques to sequence as many new organisms 
as possible, are producing a wealth of new information.
