% !TEX TS-program = pdflatex
% !TEX encoding = UTF-8 Unicode

% This is a simple template for a LaTeX document using the "article" class.
% See "book", "report", "letter" for other types of document.

\documentclass[11pt]{article} % use larger type; default would be 10pt

\usepackage[utf8]{inputenc} % set input encoding (not needed with XeLaTeX)

%%% Examples of Article customizations
% These packages are optional, depending whether you want the features they provide.
% See the LaTeX Companion or other references for full information.

%%% PAGE DIMENSIONS
\usepackage{geometry} % to change the page dimensions
\usepackage{relsize}
\geometry{a4paper} % or letterpaper (US) or a5paper or....
% \geometry{margin=2in} % for example, change the margins to 2 inches all round
% \geometry{landscape} % set up the page for landscape
%   read geometry.pdf for detailed page layout information

\usepackage{graphicx} % support the \includegraphics command and options

% \usepackage[parfill]{parskip} % Activate to begin paragraphs with an empty line rather than an indent

%%% PACKAGES
\usepackage{amsmath} 
\usepackage{amssymb} 
\usepackage{booktabs} % for much better looking tables
\usepackage{array} % for better arrays (eg matrices) in maths
\usepackage{paralist} % very flexible & customisable lists (eg. enumerate/itemize, etc.)
\usepackage{verbatim} % adds environment for commenting out blocks of text & for better verbatim
\usepackage{subfig} % make it possible to include more than one captioned figure/table in a single float
% These packages are all incorporated in the memoir class to one degree or another...

%%% HEADERS & FOOTERS
\usepackage{fancyhdr} % This should be set AFTER setting up the page geometry
\pagestyle{fancy} % options: empty , plain , fancy
\renewcommand{\headrulewidth}{0pt} % customise the layout...
\lhead{}\chead{}\rhead{}
\lfoot{}\cfoot{\thepage}\rfoot{}

%%% SECTION TITLE APPEARANCE
\usepackage{sectsty}
\allsectionsfont{\sffamily\mdseries\upshape} % (See the fntguide.pdf for font help)
% (This matches ConTeXt defaults)

%%% ToC (table of contents) APPEARANCE
\usepackage[nottoc,notlof,notlot]{tocbibind} % Put the bibliography in the ToC
\usepackage[titles,subfigure]{tocloft} % Alter the style of the Table of Contents
\renewcommand{\cftsecfont}{\rmfamily\mdseries\upshape}
\renewcommand{\cftsecpagefont}{\rmfamily\mdseries\upshape} % No bold!

%%% END Article customizations

%%% The "real" document content comes below...

\title{RNAPyro: Stacking version}
\author{Yann Ponty}
\begin{document}
\maketitle
\newcommand{\Z}[3]{\mathcal{Z}_{\substack{(#1)\\ [#3]}}^{#2}}
\newcommand{\Y}[3]{\mathcal{Y}_{\substack{(#1)\\ [#3]}}^{#2}}
\newcommand{\B}{\mathcal{B}}
\newcommand{\Kron}{\delta}
\newcommand{\ub}{\bullet}


Forward computation:
\begin{equation}
  \Z{i,j}{m}{\alpha,\beta} = \left\{
  \begin{array}{ll}
    \displaystyle
    \sum_{\substack{a'\in \B,\\ \Kron_{a',a}\le m}}  
      \Z{i+1,j}{m-\Kron_{a',a}}{a',\beta} & \text{If }S_{[i]}=\varnothing\text{ and }s_{[i]}=a\\
    \displaystyle
    \sum_{\substack{a',b'\in \B^2,\\ \Kron_{a'b',ab}\le m}}  
    \sum_{m'=0}^{m-\Kron_{a'b',ab}}      \underset{\text{If }s[i-1]=j+1}{\left(e^{\frac{-E_{\alpha \beta \to a' b'}}{RT}}\times\right)} \Z{i+1,k-1}{m-m'-\Kron_{a'b',ab}}{a',b'}\underset{\text{If }s[i-1]\neq j+1}{\left(\times\Z{k+1,j}{m'}{b',\beta}\right)} & \text{If }S_{[i]}=k, s_{[i]}=a\text{ and }s_{[k]}=b.
  \end{array}
\right.
\end{equation}

Backward computation:
\begin{equation}
  \Y{i,j}{m}{\alpha,\beta} = \left\{
  \begin{array}{ll}
    \displaystyle
    \sum_{\substack{a'\in \B,\\ \Kron_{a',a}\le m}}  
      \Y{i-1,j}{m-\Kron_{a',a}}{a',\beta} & \text{If }S_{[i]}=\varnothing\text{ and }s_{[i-1]}=a\\
    \displaystyle
    \sum_{\substack{a',b'\in \B^2,\\ \Kron_{a'b',ab}\le m}}  
    \sum_{m'=0}^{m-\Kron_{a'b',ab}}      
      e^{\frac{-E_{\alpha \beta \to a' b'}}{RT}}
      \times \Y{i-1,k+1}{m-m'-\Kron_{a'b',ab}}{a',b'}
      \times\Z{j,k-1}{m'}{\beta,b'}
    & \text{If }S_{[i]}=k, s_{[i]}=a\text{ and }s_{[k]}=b.
  \end{array}
\right.
\end{equation}

\section{First section}

Your text goes here.

\subsection{A subsection}

More text.

\end{document}
