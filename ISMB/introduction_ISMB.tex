%!TEX root = main_ISMB.tex
\section{Introduction}
\label{sec:introduction}

At the core of the emerging field of synthetic biology resides our capacity to design and re-engineer molecules with target functions. RNA molecules are well tailored for such applications. The ease to synthesize them (they are directly transcribed from DNA) and the broad diversity of catalytic and regulation functions they can perform enable to integrate \textit{de-novo} logic  circuits within living cells \cite{Rodrigo:2012fk} or re-program existing regulation mechanisms \cite{Chang:2012uq}. Future advances and applications of these techniques in gene-therapy studies will strongly rely on efficient computational methods to design and re-engineer RNA molecules.

Most of RNA functions are, at least partially, encoded by the three-dimensional molecular structures, which are themselves primarily determined by the secondary structures. The development of efficient algorithms for designing RNA sequences with pre-defined secondary structures is thus a milestone to enter the synthetic biology era. \RNAinverse pioneered RNA secondary structure design algorithms. It has been developed and distributed with the Vienna RNA package \cite{Hofacker:1994}. However, only posterior experimental studies revealed the potential and practical impact of these techniques. Thereby, during the last 6 years many improvements and variants of \RNAinverse have been proposed. Conceptually, almost all existing programs follow the same approach. First a seed sequence is selected, then a local search strategy is used to mutate the seed and find in its vicinity a sequence with desired folding properties. Using this strategy, \INFORNA \cite{Busch:2006uq}, \RNASSD \cite{Aguirre-Hernandez:2007kx} and \NUPACK \cite{Zadeh:2011fk} significantly improved the performance of the RNA secondary structure design algorithms. More recent research studies aimed to include more constraints in the selection criteria.\RNAexinv focused on the design of sequences with enhanced thermodynamical and mutational robustness \cite{Avihoo:2011fk}, while \frankenstein enables to design RNA with multiple target structures \cite{Lyngso:2012vn}.

We recently introduced with \RNAensign a novel paradigm for the search strategy of RNA secondary structure design algorithm \cite{Levin:2012kx}. Instead of a local search approach, we propose to apply a global sampling strategy of the mutational landscape using our \RNAmutants algorithm \cite{Waldispuhl2008}. This methodology offers promising performance but  suffers from a prohibitive running time complexity. Following our work, Garcia-Martin and co-workers proposed with \RNAiFOLD an alternate methodology using constraint programming techniques to prune the mutational landscape. The latter also suffers from prohibitive running times, but it is worth noting that it proposes a seed-less approach to the RNA secondary structure design problem.

In this paper we introduce \ourprog, a RNA secondary structure design algorithm that benefits of our recent algorithmic advances \cite{Reinharz:2013aa} to expand our original \RNAensign algorithm \cite{Levin:2012kx}. \ourprog addresses previous limitations of \RNAensign and offers new functionalities. First, while our previous program had a running time complexity of $\mathcal{O}(n^4)$, \ourprog has now a linear-time and space complexity that allows him to demonstrate similar speeds to any local search algorithm. Next, \ourprog is \textit{seed-less}. Unlike \RNAensign, it does not require a seed sequence to start its search. Finally, \ourprog implements a novel algorithm using weighted sampling techniques \cite{Bodini2010} that enables us to control, for the first time, \textit{explicitly} the \gc~content of the solution. This functionality is essential because wild sequences within living organisms often present medium or low \gc~content. Presumably to offer better transcription rates and/or structural plasticity. Previous programs do not allow to control this parameter and tend to output sequences with high \gc~contents. 

We demonstrate the performance of our algorithms on a set of real RNA structures extracted from the \textsf{RNA STRAND} database \cite{andronescu2008rna}. To complete this study, we develop an hybrid method combining our global sampling approach with local search strategies such as the one implemented in \RNAinverse. Remarkably, our glocal methodology overcomes both local and global approaches.