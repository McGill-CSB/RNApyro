%!TEX root = main_ISMB.tex
\section{Methods}
\label{sec:methods}



We introduce a probabilistic model for the design of RNA sequences with a specific $\Gb+\Cb$ content and folding into a predefined secondary structure.
For the sake of simplicity, we choose to base this proof-of-concept implementation on a simplified free-energy function $\ES(\cdot)$, which only considers the contributions of 
stacked canonical base-pairs. 

We show how a modification of the dynamic programming scheme introduced in the previously contributed\RNAmutants allows for the sampling of good and diverse design candidates, in linear time and space complexities.


%To that purpose, a Boltzmann weighted distribution is used, based on a pseudo-energy function $\PE{\cdot}$ which includes contributions for both the free-energy and its putative isostericity towards a multiple sequence alignment. In this model, the probability that the nucleotide at a given position needs to be mutated (i.e. corresponds to a sequencing error) can be computed using a variant of the \emph{Inside-Outside algorithm}~\cite{Lari1990}.
%
%\subsection{Probabilistic model}
%Let $\Omega$ be an gap-free RNA alignment sequence, $S$ its associated secondary structure, 
%then any sequence $s$ is assigned a probability proportional to its Boltzmann factor
%\begin{align*}
%  \mathcal{B}(s) &= e^\frac{-\PE{s}}{RT}, &&\text{with}&\PE{s}&:=\alpha\cdot\ES(s,S)+(1-\alpha)\cdot\EI(s,S,\Omega),
%\end{align*}
%where $R$ is the Boltzmann constant, $T$ the temperature in Kelvin, $\ES(s)$ and $\EI(s,S,\Omega)$ 
%are the free-energy and isostericity contributions respectively (further described below), and $\alpha\in[0,1]$ is an arbitrary parameter that sets the relative weight for both contributions.
%
%\subsubsection{Energy contribution}
\subsection{Definitions}

%\subsubsection{Profile}
%At every position, the nucleotide distribution is the same. 
%It is first initialized equiprobably. A bisection scheme is then applied until the sampled sequences $\Cb+\Gb$ content is as desired. The probability
%of a given nucleotide $a\in\B$ will be given by $\Prob(a)$ at every 
%position.

\subsubsection{Secondary structure}
Todo...

\subsubsection{Energy model}
The free-energy contribution in our pseudo-energy model corresponds to an additive stacking-pairs model, using individual values from the Turner 2004 model (retrieved from the NNDB~\cite{Turner2010}). Given a candidate sequence $s$ for a secondary structure $S$, the free-energy of $S$ on $s$ is given by
\begin{align*}
  \ES(s,S) = \sum_{\substack{(i,j)\to (i',j')\in S\\ \text{stacking pairs}}}\ES^{\beta}_{s_is_j\to s_{i'}s_{j'}} 
\end{align*}
where $\ES^{\beta}_{ab\to a'b'}$ is set to $0$ if $ab=\varnothing$ (no base-pair to stack onto), the tabulated free-energy of stacking pairs $(ab)/(a'b')$ in the Turner model if available, or $\beta\in[0,\infty]$ for non-Watson-Crick/Wobble entries (i.e. not in $\{\Gb\Ub,\Ub\Gb,\Cb\Gb,\Gb\Cb, \Ab\Ub\text{ or }\Ub\Ab\}$). This latter parameter allows one to choose whether to simply penalize invalid base pairs ($\beta>0$), or forbid them altogether ($\beta = \infty$).

\subsection{\GCContent weighted pseudo-Boltzmann ensemble and distribution}

In order to counterbalance the documented tendency of sampling methods to generate \Gb\Cb-rich sequences~\cite{Levin:2012kx}, let us introduce a parameter $x\in\mathbb{R}^+$, whose value will influence the \GCContent of generated sequences. Let $\Target$ be a targeted secondary structure and $\gc(s)$ be the number of occurrences of \Gb and \Cb in an RNA sequence $s$, then let us define the pseudo-Boltzmann factor $\B_{x}(s,S)$ of a sequence $s$ such that
\begin{equation}
\B_{x, \Target}(s) = e^{\frac{-\ES(s,\Target)}{RT}}\cdot x^{\gc(s)}
\label{def:genBoltz}
\end{equation}
where $R$ is the Boltzmann constant and $T$ the temperature in Kelvin.

Summing the pseudo-Boltzmann factor over all possible sequences of a given length $|\Target|$ gives the pseudo-partition function $\mathcal{Z}_{\Target,x}$, from which one defines the pseudo-Boltzmann probability $\Prob_{x,\Target}(s)$ of each sequence $s$, respectively such that 
\begin{align}\mathcal{Z}_{x,\Target} &= \sum_{\substack{|s|=|\Target|}}\B_{x,\Target}(s)& \text{and}&&
\Prob_{x,\Target}(s) &= \frac{\B_{x,\Target}(s)}{\mathcal{Z}_{x,\Target}}.\label{def:distribution}\end{align}

\subsection{Sampling the pseudo-Boltzmann ensemble}

Let us now describe a linear-time algorithm to sample sequences at random in the pseudo-Boltzmann distribution. This algorithm follows the general principles of the recursive approach to random generation~\cite{Wilf1977}, pioneered in the context of RNA by the \SFold algorithm~\cite{Ding2003}. The algorithm starts by precomputing the partition function restricted to each valid interval, and then performs a series of recursive stochastic backtracks, using precomputed values to decide on the probability of each alternative.

\subsubsection{Precomputing the pseudo-partition function}\label{sec:pf}
Firstly, a dynamic programming algorithm computes $\mathcal{Z}\substack{(i,j)\\ [a,b]}$ the pseudo-partition function restricted to an interval $[i,j]$, assuming its (previously chosen) flanking nucleotides are $a$ and $b$ respectively. 
Since the only sequence over an empty interval is the empty sequence, having energy $0$, one has
\begin{equation}
	\forall i \in [0,n-1]:\, \Z{i+1,i}{a,b}=1.
	\label{eq:Z_in}
\end{equation}


The recursion consists in four terms, depending on the base-pairing status and context of $(i,j)$:
\begin{equation}
	\Z{i,j}{a,b}:=\left\{
  \begin{array}{ll}
  		\displaystyle
      \sum_{a'\in \B}  
      x^{\gc(a')}
      \cdot\Z{i+1,j}{a',b} &\begin{array}{@{}l@{}}\text{If }\Target_{i}=-1\\ \text{\relsize{-1}Position $i$ unpaired,}\end{array}\\
      \displaystyle
      \sum_{a',b'\in \B^2}
			 x^{\gc(a'.b')}
			 \cdot e^{\frac{-\ES^{\beta}_{ab \to a'b'}}{RT}}
			 \cdot \Z{i+1,j-1}{a',b'}&
\begin{array}{@{}l@{}}\text{Elif }\Target_i=j \land \Target_{i-1}=j+1\\ \text{\relsize{-1}$(i,j)$ stacking pair onto $(i$-$1,j$+$1)$,}\end{array}
			 \\
			 \displaystyle
      \sum_{a',b'\in \B^2}
      x^{\gc(a'.b')}
			\cdot e^{\frac{-\ES^{\beta}_{\varnothing\to a'b'}}{RT}}
      \cdot\Z{i+1,k-1}{a',b'}
      \cdot\Z{k+1,j}{b',b} & 
\begin{array}{@{}l@{}}\text{Elif }\Target_i=k \land i < k \leq j\\
\text{\relsize{-1}Junction/bulge $\Rightarrow$ split, no stack,}
\end{array}
\\
      0 &\text{Otherwise.}
	\end{array}\right.
\label{eq:Z_rec}
\end{equation}

Despite its apparent quadratic complexity, the recurrence described in Subsection~\ref{sec:pf} can be computed in linear $\Theta(|\Target|)$ time and space, owing to the fact that, on any recursive call, the position $j$ is entirely determined by $i$. 

\begin{theorem}Starting from the complete region $[0,n-1]$, Equation~\eqref{eq:Z_rec} can be computed in $\Theta(|\Target|)$ time and space complexity.
\end{theorem}
\begin{proof}

Consider the {\bf covering base-pair} $\CNBP{i}$ of a position $i$, defined as the base-pair $(i',j')$ such that $i'<i<j'$ and which minimizes $i-i'$. For the sake of simplicity, and without loss of generality, let us assume that there exists a virtual base-pair $(-1,n)$ in $\Target$, so that any position in $[0,n-1]$ has a covering base-pair.
Then one easily shows by induction that, starting from the complete interval $[0,n-1]$, the recursive calls only evaluate $\mathcal{Z}$ over intervals $[i,j]$ such that $j=j'-1$, $\CNBP{i}=[i',j']$. 

\begin{cframed}This property clearly holds initially for the complete interval $[0,n-1]$, owing to the virtual base-pair $(-1,n)$. Furthermore, if it holds for a given $(i,j)$, $\CNBP{i}=[i',j+1]$, then:
\begin{itemize} 
  \item If $i$ is unpaired (Case 1), then $\CNBP{i+1}=\CNBP{i}=[i',j+1]$, and the property is satisfied by the subsequent call over $[i+1,j]$. 
  \item If $i$ is paired with $j$ and stacking onto $(i-1,j+1)$ (Case 2), then one has $\CNBP{i+1}=[i,j]$ and the  subsequent call over $[i+1,j-1]$ satisfies the property.
  \item If $i$ is paired with some $k$ (Case 3), then $\CNBP{i+1}=[i,k]$ and $\CNBP{k+1}=[i',j+1]$. Therefore the subsequent calls, respectively over $[i+1,k-1]$ and $[k+1,j]$, both satisfy the property.
\end{itemize}
The case of an empty sequence does not cause a recursive call and must not be verified.
\end{cframed}
Since, in any interval $[i,j]$ satisfying the property, the ending position $j$ is completely determined by $i$, then there exists at most $\Theta(|\Target|)$ intervals that need to be evaluated, and we conclude on the overall complexity with the remark that $a$, $b$, $a'$ and $b'$ vary over bounded ranges and only contribute to a constant in the complexity.
\end{proof}


\subsubsection{Stochastic backtrack}
Once the pseudo-partition functions have been computed, a stochastic backtrack starts from the complete interval $[0,|\Target|-1]$ and, at each step, chooses a suitable assignment for one or several positions, using probabilities derived from the precomputation. One or several recursive calls over regions that are left unassigned are then performed, as described by Algorithm~\ref{alg:back}.

Since, during each recursive call, the algorithm assigns at least one nucleotide to a -- previously unassigned -- position, and that the number of executions of the loops are bounded by a constant, then the complexity of the algorithm grows linearly on the targeted length. 
\begin{algorithm}[t]
\DontPrintSemicolon
	\SetAlgoLined
\SetKwFunction{Backtrack}{Backtrack}
\SetKwFunction{Random}{Random}
	rand $\leftarrow$ \Random$\left(\Z{i,j}{a,b}\right)$\tcp*[r]{Draw random number in $[0,\Z{i,j}{a,b}[$}
 \lIf(\tcp*[f]{Empty sequence}){$j<i$}{\Return{$\varepsilon$}}
	\Else{
  k $\leftarrow S_i$\;
	\If(\tcp*[f]{Unpaired}){$(k = -1)$}{
		\For{$a'\in\B$}{
			rand $\leftarrow$ rand $- x^{\gc(a')}\cdot \Z{i+1,j}{a',b}$\;
			\lIf{$\text{rand}<0$}\Return{$a'.\Backtrack\left(i+1,j,a',b,\Target\right)$}\;
		}
	}
	\lElse{
		\If(\tcp*[f]{Stacking pairs}){$(k=j) \land (S_{i-1}=j+1)$}{
			\For{$(a',b')\in\B\times\B$}{
				rand $\leftarrow$ rand $-
			 	x^{\gc(a'.b')}
			 	\cdot e^{\frac{-\ES^{\beta}_{ab \to a'b'}}{RT}}
			 	\cdot \Z{i+1,j-1}{a',b'}	$\;
				\lIf{$\text{rand}<0$}{
					\Return{$a'.\Backtrack\left(i+1,j-1,a',b',\Target\right). b'$}\;		
				}
			}
		}
		\lElse{
			\If(\tcp*[f]{Paired, not stacked}){$(\Target_i=k) \land (i < k \leq j)$}{
				\For{$(a',b')\in\B\times\B$}{
					rand $\leftarrow$ rand $-	
       x^{\gc(a'.b')}
					\cdot e^{\frac{-\ES^{\beta}_{\varnothing\to a'b'}}{RT}}
  	    	\cdot\Z{i+1,k-1}{a',b'}
	    	  \cdot\Z{k+1,j}{b',b}$\;	
 					\If{$\text{rand}<0$}{
						\Return{$a'
						.\Backtrack\left(i+1,k-1,a',b',\Target\right)
						.b'
						.\Backtrack\left(k+1,j,b',b,\Target\right)
						$}\;	
					}
				}
		 }
		}
	}
 }
\caption{\protect\Backtrack$\left(i,j,a,b,\Target\right)$\label{alg:back}}
\end{algorithm}

\subsubsection{Rejecting unsuitable candidates}

Note that, although the parameter $x$ is introduced to match the expected \GCContent with the targeted one, nothing prevents Algorithm~\ref{alg:back} to generate sequences of arbitrary \GCContent. Therefore, we use a rejection-based approach~\cite{Bodini2010} previously used by the authors in a similar context~\cite{Waldispuhl2011}.

It was established~\cite{Waldispuhl2011} that, for each value of $x$, there exists constants $\mu_x$ and $\sigma_x$ such that the \GCContent distribution observed in sequences sampled with respect to the above distribution converges asymptotically towards a normal law having expectation in $\mu\cdot n\cdot(1+o(1))$ and standard deviation in $\sigma\cdot\sqrt{n}\cdot(1+o(1))$.
Furthermore, it can easily be shown that the expected \GCContent is a continuous and strictly increasing monotonic  function of $x$, whose limits are $0$ when $x=0$ and $n$ when $x\to +\infty$. 

Consequently, for any targeted \GCContent $gc\in[0\%,100\%]$, there exists a unique value $x_{gc}$ such that generated sequences feature, on the average, the right \GCContent. Such a value can be determined using any numerical recipe, and a simple bisection procedure, further described in earlier work~\cite{Waldispuhl2011}, is used in our implementation. Furthermore, the concentration of the distribution, as asserted by the limited standard deviation, implies a growth in $\Theta(\sqrt{n})$ (resp. $\Theta(1)$, i.e. constant) for the expected number of attempts before generating a sequence having the exact targeted \GCContent  (resp. within $[gc-\kappa,gc+\kappa]$, where $\kappa\in\Theta(1/\sqrt n)$).



\subsection{Redesigning unpaired regions: A local/global hybrid approach}





%\subsubsection{Outside}	
%\begin{figure}
%\resizebox{\textwidth}{!}{%!TEX root = main_RECOMB.tex

\begin{tikzpicture}
  \definecolor{rougeForb}{HTML}{eb23238f}
  \definecolor{rougeForbP}{HTML}{6d1515ff}

  \newcommand{\BSep}{9pt}
  \newcommand{\HSep}{350pt}
  \newcommand{\RelPosA}{0pt}
  \newcommand{\RelPosB}{-130pt}
  \newcommand{\RelPosC}{-260pt}
  \newcommand{\RelPosD}{-390pt}
  \newcommand{\FitSep}{3.5pt}

  \newcommand{\LabSepB}{15pt}


  \newcommand{\CaptionTxtA}{{\bf Case 1}: Next leftward position is unpaired.}
  \newcommand{\CaptionTxtB}{{\bf Case 2}: Paired extremal positions, nesting another base-pair, forming a stacking base-pair.}
  \newcommand{\CaptionTxtC}{{\bf Case 3}: Next leftward position is paired to the right, but no stacking pairs.}
  \newcommand{\CaptionTxtD}{{\bf Case 4}: Next leftward position is paired to the left.}


  \tikzstyle{caption}=[%fill=gray!20,draw=gray!60,thick,inner sep=4pt,rounded corners=6pt,
font=\relsize{+3}\sffamily,anchor=north west,xshift=-40pt]


  \tikzstyle{basebase}=[circle,draw,thick,inner sep=0,minimum width=18pt,fill=white,font=\relsize{+2}]

  \tikzstyle{base}=[basebase]
  \tikzstyle{basesmall}=[basebase,minimum width=10pt]
  \tikzstyle{basephantom}=[basebase,dashed]
  \tikzstyle{linez}=[draw,snake=zigzag, segment aspect=.2,%
line after snake=0pt,  
        segment length=10pt,thick]
  \tikzstyle{lined}=[linez,draw,snake=none,thick]
  \tikzstyle{line}=[linez,draw,snake=none,thick]
  \tikzstyle{lineh}=[linez]
  \tikzstyle{bp}=[in=90,out=90,draw,line width=1.5pt,blue,looseness=1.7]
  \tikzstyle{blockin}=[trapezium,trapezium angle=83,  fill=blue!20, draw=blue!20!gray,line width=1.5pt, inner sep=0,drop shadow]
  \tikzstyle{blockout}=[blockin,draw=red!80!white!55!gray,fill=red!40!white!95!gray,line width=1.5pt, drop shadow]
  \tikzstyle{lbl}=[inner sep=0,font=\relsize{+3}]
  \tikzstyle{arr}=[-open triangle 60,line width=1.5pt]


 %%%%%%% Unpaired %%%%%%%
  \begin{scope}[yshift=\RelPosA]
 %%%%%%% LHS %%%%%%%
  \begin{scope}[xshift=-\HSep]
  \node[basesmall] (n-beg) at (0,0) {};
  \node[basesmall] (a-0b) at (1.4,0) {};
  \node[base] (a-0) at (2,0) {$a$};
  \node[basesmall] (a-3) at (6,0) {};
  \node[basesmall] (n-end) at (8,0) {};
  \node[right=\BSep of a-0, basephantom] (a-1) {$b$};
  \node[left=\BSep of a-3, basephantom] (a-2) {$c$};
  \path[lineh] (a-1) --  node[lbl,pos=.5,above=45pt] (lbl1) {$m$}  (a-2);
  \path[lined] (a-0) -- (a-1);
  \path[lined] (a-2) -- (a-3);
  \path[linez] (n-beg) -- (a-0b);
  \path[lined] (a-0b) -- (a-0);
  \path[linez] (a-3) -- (n-end);

  \node[right=5pt of n-end] (x) {};



  \begin{pgfonlayer}{background}
  \node[rectangle,inner sep=\FitSep,draw,fit=(n-beg)(a-0.east)] (r1) {};
  \node[rectangle,inner sep=\FitSep,draw,fit=(n-end)(a-3.west)] (r2) {};
  \path[blockout]   (r1.south west) to (r1.south east) to (r1.north east) to[out=90,in=90,looseness=0.8] (r2.north west) to (r2.south west) to (r2.south east) to (r2.north east) to[out=90,in=90,looseness=0.9] (r1.north west) to (r1.south west) ;
  \end{pgfonlayer}{background}

\node[below= \LabSepB of n-beg, caption] {\CaptionTxtA};

  \end{scope}
 %%%%%%% /LHS %%%%%%%



  \node[basesmall] (n-beg) at (0,0) {};
  \node[base, drop shadow] (a-0) at (2,0) {\color{StressColor}$a'$};
  \node[basesmall,left=\BSep of a-0] (a-0b) {};
  \node[basesmall] (a-3) at (6,0) {};
  \node[basesmall] (n-end) at (8,0) {};
  \node[right=\BSep of a-0, basephantom] (a-1) {$b$};
  \node[left=\BSep of a-3, basephantom] (a-2) {$c$};
  \path[lineh] (a-1) --  node[lbl,pos=.5,above=40pt] (lbl1) {$m-{\color{StressColor}\delta_{a,a'}}$}  (a-2);
  \path[lined] (a-0) -- (a-1);
  \path[lined] (a-2) -- (a-3);
  \path[lined] (a-0) -- (a-0b);
  \path[linez] (n-beg) -- (a-0b);
  \path[linez] (a-3) -- (n-end);



  \node[left=5pt of n-beg] (y1) {};

  \path (x) edge[arr]    (y1);

  \begin{pgfonlayer}{background}
  \node[rectangle,inner sep=\FitSep,draw,fit=(n-beg)(a-0b.east)] (r1) {};
  \node[rectangle,inner sep=\FitSep,draw,fit=(n-end)(a-3.west)] (r2) {};
  \path[blockout]   (r1.south west) to (r1.south east) to (r1.north east) to[out=90,in=90,looseness=0.8] (r2.north west) to (r2.south west) to (r2.south east) to (r2.north east) to[out=90,in=90,looseness=0.9] (r1.north west) to (r1.south west) ;
  \end{pgfonlayer}{background}
  \end{scope}



 %%%%%%% /Unpaired %%%%%%%


 %%%%%%% Stacking %%%%%%%
  \begin{scope}[yshift=\RelPosB]

 %%%%%%% LHS %%%%%%%
  \begin{scope}[xshift=-\HSep]
  \node[basesmall] (n-beg) at (0,0) {};
  \node[basesmall] (a-0b) at (1.4,0) {};
  \node[base] (a-0) at (2,0) {$a$};
  \node[base] (a-3) at (6,0) {$d$};
  \node[basesmall] (a-3b) at (6.6,0) {};
  \node[basesmall] (n-end) at (8,0) {};
  \node[right=\BSep of a-0, basephantom] (a-1) {$b$};
  \node[left=\BSep of a-3, basephantom] (a-2) {$c$};
  \path[lineh] (a-1) --  node[lbl,pos=.5,above=45pt] (lbl1) {$m$}  (a-2);
  \path[lined] (a-0) -- (a-1);
  \path[lined] (a-2) -- (a-3);
  \path[linez] (n-beg) -- (a-0b);
  \path[lined] (a-0b) -- (a-0);
  \path[lined] (a-3) -- (a-3b);
  \path[linez] (a-3b) -- (n-end);

  \node[right=5pt of n-end] (x) {};

  \draw[bp,looseness=.9] (a-0) to (a-3);
  \draw[bp,looseness=.9] (a-1) to (a-2);


  \begin{pgfonlayer}{background}
  \node[rectangle,inner sep=\FitSep,draw,fit=(n-beg)(a-0.east)] (r1) {};
  \node[rectangle,inner sep=\FitSep,draw,fit=(n-end)(a-3.west)] (r2) {};
  \path[blockout]   (r1.south west) to (r1.south east) to (r1.north east) to[out=90,in=90,looseness=0.8] (r2.north west) to (r2.south west) to (r2.south east) to (r2.north east) to[out=90,in=90,looseness=0.9] (r1.north west) to (r1.south west) ;
  \end{pgfonlayer}{background}
  \end{scope}

\node[below= \LabSepB of n-beg, caption] {\CaptionTxtB};

 %%%%%%% /LHS %%%%%%%


  \node[basesmall] (n-beg) at (0,0) {};
  \node[base, drop shadow] (a-0) at (2,0) {\color{StressColor}$a'$};
  \node[basesmall,left=\BSep of a-0] (a-0b) {};
  \node[base, drop shadow] (a-3) at (6,0) {\color{StressColor}$d'$};
  \node[basesmall,right=\BSep of a-3] (a-3b) {};
  \node[basesmall] (n-end) at (8,0) {};
  \node[right=\BSep of a-0, basephantom] (a-1) {$b$};
  \node[left=\BSep of a-3, basephantom] (a-2) {$c$};
  \path[lineh] (a-1) --  node[lbl,pos=.5,above=45pt] (lbl1) {$m-{\color{StressColor}\delta_{ad,a'd'}}$}  (a-2);
  \path[lined] (a-0) -- (a-1);
  \path[lined] (a-2) -- (a-3);
  \path[lined] (a-0) -- (a-0b);
  \path[lined] (a-3b) -- (a-3);
  \path[linez] (n-beg) -- (a-0b);
  \path[linez] (a-3b) -- (n-end);
  \draw[bp,looseness=.8] (a-1) to (a-2);
  \draw[bp,looseness=.8] (a-0) to (a-3);




  \node[left=5pt of n-beg] (y2) {};

  \path (x) edge[arr]    (y2);


  \begin{pgfonlayer}{background}
  \node[rectangle,inner sep=\FitSep,draw,fit=(n-beg)(a-0b.east)] (r1) {};
  \node[rectangle,inner sep=\FitSep,draw,fit=(n-end)(a-3b.west) ] (r2) {};
  \path[blockout]   (r1.south west) to (r1.south east) to (r1.north east) to[out=90,in=90,looseness=0.8] (r2.north west) to (r2.south west) to (r2.south east) to (r2.north east) to[out=90,in=90,looseness=0.9] (r1.north west) to (r1.south west) ;
  \end{pgfonlayer}{background}
  \end{scope}
 %%%%%%% /Stacking %%%%%%%

 %%%%%%% BPRight %%%%%%%


  \begin{scope}[yshift=\RelPosC]

 %%%%%%% LHS %%%%%%%
  \begin{scope}[xshift=-\HSep]
  \node[basesmall] (n-beg) at (0,0) {};
  \node[base] (a-0) at (1.3,0) {$a$};
  \node[basesmall] (a-3) at (4.5,0) {};
  \node[basesmall] (a-4b) at (5.8,0) {};
  \node[base] (a-4) at (6.4,0) {$d$};
  \node[basesmall] (a-4t) at (7,0) {};
  \node[basesmall] (n-end) at (8,0) {};
  \node[right=\BSep of a-0, basephantom] (a-1) {$b$};
  \node[left=\BSep of a-3, basephantom] (a-2) {$c$};
  \path (n-beg) --  node[lbl,pos=.5,above=45pt] (lbl1) {$m$}  (n-end);
  \path[lineh] (a-1) --   (a-2);
  \path[lined] (a-0) -- (a-1);
  \path[lined] (a-2) -- (a-3);
  \path[linez] (n-beg) -- (a-0);
  \path[linez] (a-3) -- (a-4b);
  \path[line] (a-4b) -- (a-4);
  \path[line] (a-4) -- (a-4t);
  \path[linez] (a-4t) -- (n-end);

  \draw[bp,looseness=.7] (a-0) to (a-4);
  \draw[bp,looseness=.9,dashed] (a-1) to (a-2);


  \node[right=5pt of n-end] (x) {};

  \begin{pgfonlayer}{background}
  \node[rectangle,inner sep=\FitSep,draw,fit=(n-beg)(a-0.east)] (r1) {};
  \node[rectangle,inner sep=\FitSep,draw,fit=(n-end)(a-3.west) ] (r2) {};
  \path[blockout]   (r1.south west) to (r1.south east) to (r1.north east) to[out=90,in=90,looseness=0.8] (r2.north west) to (r2.south west) to (r2.south east) to (r2.north east) to[out=90,in=90,looseness=0.9] (r1.north west) to (r1.south west) ;
  \end{pgfonlayer}{background}
  \end{scope}

\node[below= \LabSepB of n-beg, caption] {\CaptionTxtC};

 %%%%%%% /LHS %%%%%%%

  \node[basesmall] (n-beg) at (0,0) {};
  \node[base,drop shadow] (a-0) at (1.8,0) {\color{StressColor}$a'$};
  \node[basesmall,left=\BSep of a-0] (a-0b) {};
  \node[base,drop shadow] (a-3) at (6.25,0) {\color{StressColor}$d'$};
  \node[basesmall] (b-1) at (4.0,0) {};
  \node[basesmall,left=\BSep of a-3] (b-2){};
  \node[basesmall,right=\BSep of a-3] (a-3b) {};
  \node[basesmall] (n-end) at (8,0) {};
  %\node[right=\BSep of a-0, basephantom] (a-1) {$a$};
  \node[left=\BSep of b-1, basephantom] (a-2) {$c$};
  \path[lineh] (a-0) --   (a-2);
  \path[lined] (a-2) -- (b-1);
  \path[linez] (b-1) -- node[lbl,pos=.5,above=8pt] (lbl1) {$m'$}  (b-2);
  \path (n-beg) -- node[lbl,pos=.5,above=46pt] (lbl1) {$m-m'-{\color{StressColor}\delta_{ad,a'd'}}$}  (n-end);
  \path[lined] (b-2) -- (a-3);
  \path[lined] (a-0) -- (a-0b);
  \path[lined] (a-3b) -- (a-3);
  \path[linez] (n-beg) -- (a-0b);
  \path[linez] (a-3b) -- (n-end);
  \draw[bp,looseness=.8] (a-0) to (a-3);

  \node[left=5pt of n-beg] (y3) {};
  \path (x) edge[arr]    (y3);

  \begin{pgfonlayer}{background}
  \node[rectangle,inner sep=\FitSep,draw,fit=(n-beg)(a-0b.east)] (r1) {};
  \node[rectangle,inner sep=\FitSep,draw,fit=(n-end)(a-3b.west) ] (r2) {};
  \path[blockout]   (r1.south west) to (r1.south east) to (r1.north east) to[out=90,in=90,looseness=0.8] (r2.north west) to (r2.south west) to (r2.south east) to (r2.north east) to[out=90,in=90,looseness=0.9] (r1.north west) to (r1.south west) ;

  \node[rectangle,inner sep=\FitSep,draw,fit=(b-1)(b-2) ] (r3) {};
  \path[blockin]   (r3.south west) to (r3.south east) to (r3.north east) to[out=90,in=90,looseness=1.4] (r3.north west) to (r3.south west) ;
  \end{pgfonlayer}{background}
  \end{scope}
 %%%%%%% /BPRight %%%%%%%


 %%%%%%% BP Left %%%%%%%
  \begin{scope}[yshift=\RelPosD]

 %%%%%%% LHS %%%%%%%
  \begin{scope}[xshift=-\HSep]
  \node[basesmall] (n-beg) at (0,0) {};
  \node[basesmall] (a-w) at (1.1,0) {};
  \node[base] (a-x) at (1.7,0) {$a$};
  \node[basesmall] (a-y) at (2.3,0) {};
  \node[basesmall] (a-z) at (3.3,0) {};
  \node[base] (a-0) at (3.9,0) {$b$};
  \node[basesmall] (a-3) at (7,0) {};
  \node[basesmall] (n-end) at (8,0) {};
  \node[right=\BSep of a-0, basephantom] (a-1) {$c$};
  \node[left=\BSep of a-3, basephantom] (a-2) {$d$};
  \path[lineh] (a-1) --  (a-2);
  \path (n-beg) --  node[lbl,pos=.5,above=45pt] (lbl1) {$m$}  (n-end);
  \path[lined] (a-0) -- (a-1);
  \path[lined] (a-2) -- (a-3);
  \path[linez] (n-beg) -- (a-w);
  \path[line] (a-w) -- (a-x);
  \path[linez] (a-x) -- (a-y);
  \path[linez] (a-y) -- (a-z);
  \path[line] (a-z) -- (a-0);
  \path[linez] (a-3) -- (n-end);

  \node[right=5pt of n-end] (x) {};

  \draw[bp,looseness=1.05] (a-x) to (a-0);
  \draw[bp,looseness=.9,dashed] (a-1) to (a-2);


  \begin{pgfonlayer}{background}
  \node[rectangle,inner sep=\FitSep,draw,fit=(n-beg)(a-0.east)] (r1) {};
  \node[rectangle,inner sep=\FitSep,draw,fit=(n-end)(a-3.west) ] (r2) {};
  \path[blockout]   (r1.south west) to (r1.south east) to (r1.north east) to[out=90,in=90,looseness=0.8] (r2.north west) to (r2.south west) to (r2.south east) to (r2.north east) to[out=90,in=90,looseness=0.9] (r1.north west) to (r1.south west) ;
  \end{pgfonlayer}{background}
  \end{scope}
 
\node[below= \LabSepB of n-beg, caption] {\CaptionTxtD};


%%%%%%% /LHS %%%%%%%

  \node[basesmall] (n-beg) at (0,0) {};
  \node[base, drop shadow] (a-0) at (1.8,0) {\color{StressColor}$a'$};
  \node[basesmall,left=\BSep of a-0] (a-0b) {};
  \node[base, drop shadow] (a-3) at (4.5,0) {\color{StressColor}$b'$};
  \node[basesmall,right=\BSep of a-0] (b-1)  {};
  \node[basesmall,left=\BSep of a-3] (b-2){};
  \node[basesmall] (a-3b)  at (6.75,0) {};
  \node[basephantom,left=\BSep of a-3b] (a-3c){$d$};
  \node[basesmall] (n-end) at (8,0) {};


  %\node[right=\BSep of a-0, basephantom] (a-1) {$a$};
  %\node[left=\BSep of b-1, basephantom] (a-2) {$b$};
  \path[lined] (b-1) -- (a-0);
  \path[linez] (b-1) -- node[lbl,pos=.5,above=8pt] (lbl1) {$m'$}  (b-2);
  \path (n-beg) -- node[lbl,pos=.5,above=43pt] (lbl1) {$m-m'-{\color{StressColor} \delta_{ab,a'b'}}$}  (n-end);
  \path[lined] (a-3b) -- (a-3c);
  \path[lined] (a-0) -- (a-0b);
  \path[lined] (b-2) -- (a-3);
  \path[lineh] (a-3c) -- (a-3);
  \path[linez] (n-beg) -- (a-0b);
  \path[linez] (a-3b) -- (n-end);
  \draw[bp,looseness=1.05] (a-0) to (a-3);

  \node[left=5pt of n-beg] (y4) {};
  \path (x) edge[arr]    (y4);


  \begin{pgfonlayer}{background}
  \node[rectangle,inner sep=\FitSep,draw,fit=(n-beg)(a-0b.east)] (r1) {};
  \node[rectangle,inner sep=\FitSep,draw,fit=(n-end)(a-3b.west) ] (r2) {};
  \path[blockout]   (r1.south west) to (r1.south east) to (r1.north east) to[out=90,in=90,looseness=0.8] (r2.north west) to (r2.south west) to (r2.south east) to (r2.north east) to[out=90,in=90,looseness=0.9] (r1.north west) to (r1.south west) ;

  \node[rectangle,inner sep=\FitSep,draw,fit=(b-1)(b-2)] (r3) {};
  \path[blockin]   (r3.south west) to (r3.south east) to (r3.north east) to[out=90,in=90,looseness=1.4] (r3.north west) to (r3.south west) ;
  \end{pgfonlayer}{background}
  \end{scope}
 %%%%%%% /BP Left %%%%%%%

\end{tikzpicture}}
%\caption{Principle of the outside computation. Note that the outside algorithm uses intermediate results from the inside algorithm, 
%therefore its efficient implementation requires an implementation of the inside computation.}
%\end{figure}
%The \emph{Outside} function, $\mathcal Y$, is the partition function considering only the 
%contributions of subsequences $[0,i]\cup[j,n-1]$ over the mutants of $s$ having exactly $m$ mutations between $[0,i]\cup[j,n-1]$ and whose nucleotide at position $i+1$ is $a$ 
%(resp. in position $j-1$ it is $b$).
%We define function $\Y{i,j}{m}{a,b}$ as a recurrence, and will use as initial conditions:
%\begin{equation}
%	\Y{-1,j}{m}{X,X}:=
%		\displaystyle
%	  \Z{j,n-1}{m}{X,X}
%\label{eq:Y_in}
%\end{equation}
%The recurrence, as shown below, will increase the interval $[i,j]$ by decreasing $i$ when
%it is not base paired. If it is with a position $k>j$, we increase $j$ to include it.
% Thus, when we need
%to evaluate an interval as $(-1,j)$, all stems between $(0,j)$ are taken into account and the
%structure between $(j,n-1)$ must be a set of independent stems. Therefore,
% all the outside energy between $[j,n-1]$ is
%equal to $\Z{j,n-1}{m}{X,X}$, for any $X\in B$. The recursion itself is as follows.
%\begin{equation}
%	\Y{i,j}{m}{a,b} = \left\{
%  \begin{array}{ll}
%		\displaystyle
%    \sum_{\substack{a'\in \B,\\ \Kron_{a',s_i}\le m}}
%    \Y{i-1,j}{m- \Kron_{a',s_i}}{a',b} &
%    \text{Elif }S_i=-1 \\
%    \displaystyle
%    \sum_{\substack{a'b'\in \B^2,\\ \Kron_{a'b',s_is_j}\le m}}
%		 e^{\frac{-E_{(i,j),ab \to a'b'}^{\Omega,\beta}}{RT}}\cdot
%    \Y{i-1,j+1}{m- \Kron_{a'b',s_is_j}}{a',b'} &
%   	 \text{Elif }S_{i}=j \land S_{i+1}=j-1\\
%		 \displaystyle
%		 \sum_{\substack{a'b'\in \B^2,\\ \Kron_{a'b',s_is_k}\le m}}
%		 \sum_{m'=0}^{m-\Kron_{a'b',s_is_k}}
%  		 e^{\frac{-E_{(i,k),\varnothing\to a'b'}^{\Omega,\beta}}{RT}}
%		 \cdot\Y{i-1,k+1}{m- \Kron_{a'b',s_is_k} - m'}{a',b'}
%     \cdot\Z{j,k-1}{m'}{b,b'} &
%		 \text{Elif }S_{i}=k \geq j\\
%		 \displaystyle
%		 \sum_{\substack{a'b'\in \B^2,\\ \Kron_{a'b',s_ks_i}\le m}}
%		 \sum_{m'=0}^{m-\Kron_{a'b',s_ks_i}}
%   	 e^{\frac{-E_{(k,i),\varnothing\to a'b'}^{\Omega,\beta}}{RT}}
%		 \cdot\Y{k-1,j}{m- \Kron_{a'b',s_ks_i} - m'}{a',b}
%     \cdot\Z{k+1,i-1}{m'}{a',b'} &
%		 \text{Elif }-1 < S_{i}=k < i\\
%		 0 & \text{Otherwise}
%  \end{array}\right.
%\label{eq:Y_rec}
%\end{equation}
%The five cases can be broked down as follows.
%\begin{description}
%\item[$S_i=-1$:] If the nucleotide at position $i$ is not paired, then the value is the same
%as if we decrease the lower interval bound by $1$ (i.e. $i-1$), and consider all possible
%nucleotides $a'$ at position $i$, correcting the number of mutants
%in function of $\Kron_{a',s_i}$.
%\item[$S_{i}=j$ and $S_{i+1}=j-1$:] If nucleotide $i$ is paired with $j$ and nucleotide $i+1$ is
%paired with $j-11$, we are in the only case were stacked base pairs can occur. We thus add
%the energy of the stacking and of the isostericity of the base pair $(i,j)$. What is left
%to compute is the \emph{outside} value for the interval $[i-1,j+1]$ over all possible nucleotides 
%$a',b'\in B^2$ at positions $i$ and $j$ respectively.
%\item[$S_{i}=k \geq j$:]If nucleotide $i$ is paired with position $k\geq j$, 
%and is not stacked inside, the 
%only term contributing directly to the energy is the isostericity of the base pair $(i,k)$. 
%We can then consider the outside interval $[i-1,k+1]$ by multiplying it by the the \emph{inside}
%value of the newly included interval (i.e. $[j,k-1]$), for 
%all possible values $a',b'\in B^2$ for nucleotides at positions $i$ and $k$ respectively.
%\item[$-1<S_{i}<i$:]As above but if position $i$ is to pairing with a lower value.
%\item[Else:] In all other cases, we are in a derivation of the SCFG that does not correspond to the 
%secondary structure $S$, and we return $0$.
%
%
%\end{description}
%
%\subsubsection{Combining Inside and Outside values into point-wise mutations probabilities}
%By construction, the partition function over all sequences at exactly $m$ mutations of $s$ can 
%be described in function of the \emph{inside} term as $\Z{0,n-1}{m}{X,X}$,
% for any nucleotide $X\in B$ or
%in function of the \emph{outside} term, for any unpaired position $k$:
%$$
%	\Z{0,n-1}{m}{X,X}
%	\equiv
%	\sum_{\substack{a\in \B,\\ \Kron_{a,s[k]}\le m}}	
%	\Y{k-1,k+1}{m-\Kron_{a,s[k]}}{a,a}
%$$
%
%We are now left to compute the probability that a given position is a given nucleotide.
%We leverage the \emph{Inside-Outside} construction to immediately obtain the following $3$ cases.
%Given $i\in[0,n-1],x\in B$, and $M\geq 0$ a bound on the number of allowed mutations. 
%\begin{align}
%	\mathbb{P}(s_i = x\mid s,\Omega, S,M) &:= \frac{\mathcal{W}(i,x,s,\Omega,S,M)}{\sum_{m=0}^{M}\Z{0,n-1}{m}{X,X}}\label{eq:normalize}\\ 
%\mathcal{W}(i,x,s,\Omega,S,M)&=
% \left\{
%	\begin{array}{ll}
%			\sum_{m=0}^{M}
%			\Y{i-1,i+1}{m-\Kron_{x,s_i}}{x,x}
%		&\text{If }S_i = -1\\
%			\displaystyle
%			\sum_{m=0}^{M}
%			\sum_{\substack{b\in B\\\Kron_{xb,s_is_k\leq m}}}
%			\sum_{m'=0}^{m-\Kron_{xb,s_is_k}}
%     	 e^{\frac{-E_{(i,k),\varnothing\to xb}^{\Omega,\beta} }{RT}}
%			\cdot\Y{i-1,k+1}{m-\Kron_{xb,s_is_k-m'}}{x,b}
%			\cdot\Z{i+1,k-1}{m'}{x,b}
%		&\text{If }S_i=k>i\\
%    \displaystyle
%			\sum_{m=0}^{M}
%			\sum_{\substack{b\in B\\\Kron_{bx,s_ks_i\leq m}}}
%			\sum_{m'=0}^{m-\Kron_{bx,s_ks_i}}
%     	 e^{\frac{-E^{\Omega, \beta}_{(k,i),\varnothing\to bx}}{RT}}
%			\cdot\Y{k-1,i+1}{m-\Kron_{bx,s_ks_i-m'}}{b,x}
%			\cdot\Z{k+1,i-1}{m'}{b,x}
%		&\text{If }S_i=k<i
%	\end{array}\right.\label{eq:combine}
%\end{align}
%
%In every case, the denominator is the sum of the partitions function of exactly $m$ mutations, 
%for $m$ smaller or equal to our target $M$. The numerators are divided in the following three cases.
%\begin{description}
%\item[$S_i=-1$:] If the nucleotide at position $i$ is not paired, we are concerned by the weights
%over all sequences which have at position $i$ nucleotide $x$, which is exactly the sum
%of the values of $\Y{i-1,i+1}{m-\Kron_{x,s_i}}{x,x}$, for all $m$ between $0$ and $M$.
%\item[$S_i=k>i$:] Since we need to respect the derivation of the secondary structure $S$, if 
%position $i$ is paired, we must consider the two partition functions. The \emph{outside} of the 
%base pair, and the \emph{inside}, for all possible values for the nucleotide at position $k$, and
%all possible distribution of the mutant positions between the inside and outside of the base pair. We also add the term of isostericity for this specific base pair.
%\item[$S_i=k<i$:] Same as above, but with position $i$ pairing with a lower position.
%\end{description}
%
%\subsection{Complexity considerations}
%Equations~\eqref{eq:Z_rec} and~\eqref{eq:Y_rec} can be computed using dynamic programming. Namely, the $\mathcal{Z}^{*}_{*}$ and $\mathcal{Y}^{*}_{*}$ terms are computed starting from smaller values of $m$ and interval lengths, memorizing the results as they become available to ensure constant-time access during later stages of the computation. Furthermore, energy terms $E(\cdot)$ can be accessed in constant time thanks to a simple precomputation (not described)  of the isostericity contributions in $\Theta(n\cdot|\Omega|)$. Computing any given term therefore requires $\Theta(m)$ operations.
%
%In principle, $\Theta(m\cdot n^2)$ terms, identified by $(m,i,j)$ triplets, should be computed.
%However, a close inspection of the recurrences reveals that the computation can be safely restricted to a subset of intervals $(i,j)$.
%For instance, the inside algorithm only requires computing intervals $[i,j]$ that do not break any base-pair, and whose next position $j+1$ is either past the end of the sequence, or is base-paired prior to $i$. Similar constraints hold for the outside computation, resulting in a drastic limitation of the combinatorics of required computations, dropping from $\Theta(n^2)$ to $\Theta(n)$ the number of terms that need to be computed and stored. Consequently the overall complexity of the algorithm is $\Theta(n\cdot(|\Omega|+m^2))$ arithmetic operations and $\Theta(n\cdot(|\Omega|+m))$ memory.
