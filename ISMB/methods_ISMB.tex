%!TEX root = main_ISMB.tex
\section{Methods}
\label{sec:methods}

\newcommand{\PE}[1]{E(#1)}
\newcommand{\EI}{\text{EI}}
\newcommand{\ES}{\text{ES}}
\newcommand{\ISO}{\text{ISO}}
We introduce a probabilistic model, aiming to design RNA sequences with a 
specific $\Gb+\Cb$ content, folding into a predefined 2D conformation.
We use a Boltzmann weighted distribution based on a pseudo-energy function 
$\PE{\cdot}$ taking into accounts stacked canonical base pairs free-energy. 


%To that purpose, a Boltzmann weighted distribution is used, based on a pseudo-energy function $\PE{\cdot}$ which includes contributions for both the free-energy and its putative isostericity towards a multiple sequence alignment. In this model, the probability that the nucleotide at a given position needs to be mutated (i.e. corresponds to a sequencing error) can be computed using a variant of the \emph{Inside-Outside algorithm}~\cite{Lari1990}.

\subsection{Probabilistic model}
Let $\Omega$ be an gap-free RNA alignment sequence, $S$ its associated secondary structure, 
then any sequence $s$ is assigned a probability proportional to its Boltzmann factor
\begin{align*}
  \mathcal{B}(s) &= e^\frac{-\PE{s}}{RT}, &&\text{with}&\PE{s}&:=\alpha\cdot\ES(s,S)+(1-\alpha)\cdot\EI(s,S,\Omega),
\end{align*}
where $R$ is the Boltzmann constant, $T$ the temperature in Kelvin, $\ES(s)$ and $\EI(s,S,\Omega)$ 
are the free-energy and isostericity contributions respectively (further described below), and $\alpha\in[0,1]$ is an arbitrary parameter that sets the relative weight for both contributions.

\subsubsection{Energy contribution}
The free-energy contribution in our pseudo-energy model corresponds to an additive stacking-pairs model, taking values from the Turner 2004 model retrieved from the NNDB~\cite{Turner2010}. Given a candidate sequence $s$ for a secondary structure $S$, the free-energy of $S$ on $s$ is given by
\begin{align*}
  \ES(s,S) = \sum_{\substack{(i,j)\to (i',j')\in S\\ \text{stacking pairs}}}\ES^{\beta}_{s_is_j\to s_{i'}s_{j'}} 
\end{align*}
where $\ES^{\beta}_{ab\to a'b'}$ is set to $0$ if $ab=\varnothing$ (no base-pair to stack onto), the tabulated free-energy of stacking pairs $(ab)/(a'b')$ in the Turner model if available, or $\beta\in[0,\infty]$ for non-Watson-Crick/Wobble entries (i.e. not in $\{\Gb\Ub,\Ub\Gb,\Cb\Gb,\Gb\Cb, \Ab\Ub\text{ or }\Ub\Ab\}$). This latter parameter allows to choose whether to simply penalize invalid base pairs, or forbid them altogether ($\beta = \infty$).
The loss of precision due to this simplification of the Turner model remains reasonable since the targeted secondary structure is fixed 
(e.g. multiloops do not account for base-specific contributions). Furthermore, it greatly eases the design and implementation of dynamic-programming equations. 
\subsubsection{Isostericity contribution}
The concept of isostericity score~\cite{Stombaugh2009} is based on the geometric discrepancy (superimposability) of two base-pairs, using individual additive contributions computed by Stombaugh~\emph{et al}~\cite{Stombaugh2009}. Let $s$ be a candidate sequence for a secondary structure $S$, given in the context of a gap-free RNA alignment $\Omega$,  we define the contribution of the isostericity to the pseudo-energy as
\begin{align*}
  \ES(s,S,\Omega) &= \sum_{\substack{(i,j)\in S\\ \text{pairs}}}\EI^{\Omega}_{(i,j),s_i s_j}, & \text{where}&& 	\EI^{\Omega}_{(i,j),ab}:=
	\frac{
		\sum_{s'\in\Omega}
			\text{\ISO}((s_i',s_j'),(a,b))}
%-		\left(			\text{\ISO}((s_i',s_j'),(s_i,s_j))				\right)	
{		
		|\Omega|
	}
\end{align*}
is the average isostericity of a base-pair in the candidate sequence, compared with the reference alignment.
The $\ISO$ function uses the {Watson-Crick/Watson-Crick} cis isostericity matrix computed by Stombaugh~\emph{et al}~\cite{Stombaugh2009}. Isostericity scores range between $0$ and $9.7$, with $0$ being assigned to a perfect isostericity, and a penalty of $10$ is used for missing entries.
The isostericity contribution will favor exponentially sequences that are likely to adopt a similar local conformation as the sequences contained in the alignment.

\subsection{Mutational profile of sequences}


Let $s$ be an RNA sequence, $S$ a reference structure, and $m\geq 0$ a desired number of mutations. 
We are interested in  the probability that a given position contains a specific nucleotide, over all sequences having at most $m$ mutations from $s$
%  under the SCFG derivation $S$ 
(formally $\mathbb{P}(s_i = x\mid s,\Omega, S,m)$). 
We define a variant of the
 \emph{Inside-Outside algorithm}~\cite{Lari1990}, allowing us to obtain the
 desired probability,  the two  functions
$\mathcal{Z}_*^*$ and $\mathcal{Y}_*^*$. 

The former, defined in Equations~\eqref{eq:Z_in} and~\eqref{eq:Z_rec}, is analogous to the \emph{inside} algorithm.
It is the partition function, i.e. the sum of Boltzmann factors, over all sequences within $[i,j]$, 
knowing that position $i-1$ is composed of nucleotide $a$ (resp. $j+1$ is $b$), within 
$m$ mutations of $s$. The latter, defined by Equations~\eqref{eq:Y_in} and~\eqref{eq:Y_rec},
 computes the \emph{outside} algorithm, i.e.  
the partition function over sequences within $m$ mutations of $s$, restricted to two intervals $[0,i]\cup[j,n-1]$, and 
knowing  that position $i+1$ is composed of $a$ (resp. $j-1$ is $b$). A suitable combination of these terms, given in Equation~\eqref{eq:combine}, gives the total weight, and in turn the probability, of seeing a specific base at a given position.

%Drawing a parallel with stochastic context-free grammars (SCFG), for which the inside-outside algorithm was introduced, the set of sequences can be seen as the language of words having length $n$, generated from a stochastic context-free grammar.
%These rules would limit the number of mutations (i.e. )
% $S$ can be considered as constraining weighing on the shape of parse trees the derivations in a classic 
% of the {SCFG} generating  all  secondary structures of length  $|s|$. 





\subsubsection{Definitions}
Let $B:=\left\{\Ab,\Cb,\Gb,\Ub\right\}$ be the set of nucleotides.
Given $s\in B^n$ an RNA sequence, let $s_i$ be the nucleotide at position $i$. Let $\Omega$ be a set of un-gapped RNA sequences of
length $n$, and $S$ a secondary structure without pseudoknots. 
Formally, if $(i,j)$ and $(k,l)$ are base pairs in $S$, there is no overlapping extremities
 $\{i,j\}\cap \{k,l\}=\varnothing$ and either the intersection is empty 
 ($[i,j]\cap[k,l]=\varnothing$) or one is included in the other ($[k,l]\subset[i,j]$ or 
 $[i,j]\subset[k,l])$. 

Let us then remind a Hamming distance function $\delta: B^*\times B^* \to \mathbb{N}^+$, which takes two sequences $s'$ and $s''$ as input, $|s'|=|s''|$, and returns the number of differing positions.
Finally, let us denote by $E_{(i,j),ab \to ab'}^{\Omega,\beta}$ the local contribution of a base-pair $(i,j)$ to the pseudo-energy, such that
\begin{equation}
  E_{(i,j),ab \to a'b'}^{\Omega,\beta}  = \alpha \cdot\ES^\beta_{a b \to a' b'}+(1-\alpha)\cdot\EI^{\Omega}_{(i,j),a'b'}.
\end{equation}


\subsubsection{Inside}
\begin{figure}
\resizebox{\textwidth}{!}{
\begin{tikzpicture}[scale=.95]

  \newcommand{\BSep}{9pt}
  \newcommand{\HSep}{250pt}
  \newcommand{\VSepUp}{140pt}
  \newcommand{\VSepDown}{-140pt}
  \newcommand{\LabSepA}{63pt}
  \newcommand{\LabSepB}{75pt}
  \newcommand{\LabSepC}{72pt}
 
  \tikzstyle{base}=[circle,draw,thick,inner sep=0,minimum width=17pt,fill=white,,font={\relsize{+2}}]
  \tikzstyle{basesmall}=[circle,draw,thick,inner sep=0,minimum width=10pt,fill=white]
  \tikzstyle{basephantom}=[base,dashed,font=\relsize{+2}]
  \tikzstyle{linez}=[draw,snake=zigzag, segment aspect=.2,%
line after snake=0pt, 
        segment length=10pt,thick]
  \tikzstyle{lined}=[linez,draw,snake=none,thick]
  \tikzstyle{line}=[linez,draw,snake=none,thick]
  \tikzstyle{bp}=[in=95,out=85,draw,line width=1.5pt,blue,looseness=1]
  \tikzstyle{block}=[trapezium,trapezium angle=33, fill=blue!20, draw=blue!20!gray,line width=1.5pt, inner sep=0]
  \tikzstyle{lbl}=[inner sep=0]
  \tikzstyle{arr}=[line width=1.5pt,-open triangle 60,line width=2pt]
  \tikzstyle{caption}=[%fill=gray!20,draw=gray!60,thick,inner sep=4pt,rounded corners=6pt,
font=\relsize{+3},anchor=north west,xshift=-70pt]
  \tikzstyle{prob}=[fill=none,above=5pt,font=\relsize{+4},draw=none]
  \tikzstyle{cap2}=[fill=none,above=12pt,font=\relsize{+3}]





%% Inside part %%%%%%%%%%
{
%% Unpaired case %%%%%%%%%%

\begin{scope}[yshift=\VSepUp]
  \node[base] (a-1) at (0,0) {$b$};
  \node[basesmall] (a-1b) at (.8,0) {};
  \node[basesmall] (a-2) at (4,0) {};
  \node[left=\BSep of a-1, basephantom] (a-0) {$a$};
  \node[right=\BSep of a-2, basephantom] (a-3) {$c$}; 
  \node[right=0 of a-3] (x1) {};


  \path[lined] (a-1) --  (a-1b);
  \path[linez] (a-1b) --  (a-2);
  \path[lined] (a-0) -- (a-1);
  \path[lined] (a-2) -- (a-3);
  \node[lbl,below=3pt of a-1] (c1) {};
  \node[lbl,below=3pt of a-2] (c2) {};
  \path (a-1) --  node (xx1) {\phantom{xx3}} (a-2);

  \path[draw=none] (a-1) -- node[cap2] (s1) {$\Struct$}  (a-2);

  \begin{pgfonlayer}{background}
  \node[rectangle,inner sep=2pt,draw,fit=(a-1.west)(a-2.east) (c1) (c2)] (r1) {};
  \path[block]   (r1.south west) to (r1.south east) to (r1.north east) to[out=90,in=90,looseness=1.1] (r1.north west) to (r1.south west) ;
  \end{pgfonlayer}{background}
\end{scope}
\node[above= \LabSepA of a-1, caption] {{\bf Case 1:} First position is unpaired.};

\begin{scope}[xshift=\HSep,yshift=\VSepUp]
  \node[base] (a-1) at (0,0) {\color{red}$b'$};
  \node[basesmall,right=\BSep of a-1] (a-1b)  {};
  \node[basesmall] (a-2) at (5,0) {};
  \node[left=\BSep of a-1, basephantom] (a-0) {}; 
  \node[right=\BSep of a-2, basephantom] (a-3) {$c$};
  \path[linez] (a-1b) -- node (xy1) {} node[cap2] {$\Struct'$}  (a-2);
  \path[line] (a-1) -- (a-1b);
  \path[lined] (a-0) -- (a-1);
  \path[lined] (a-2) -- (a-3);
  \node[below=3pt of a-1b, inner sep=0] (c1) {};
  \node[below=3pt of a-2, inner sep=0] (c2) {};

  \node[xshift=-2pt] at (a-0.west) (y1) {};

\begin{pgfonlayer}{background}
  \node[rectangle,inner sep=2pt,draw,fit=(a-1b.west)(a-2.east) (c1) (c2)] (r1) {};
  \path[block]   (r1.south west) to  (r1.south east) to (r1.north east) to[out=90,in=90,looseness=1] (r1.north west) to (r1.south west) ;
  \end{pgfonlayer}{background}

\end{scope}

  \path (x1) edge[arr]    (y1);


%%%Stacking case %%%
\begin{scope}[yshift=0]
  \node[base] (a-1) at (0,0) {$b$};
  \node[base] (a-2) at (4,0) {$c$};


  \node[left=\BSep of a-1, basephantom] (a-0) {$a$};
  \node[right=\BSep of a-2, basephantom] (a-3) {$d$};
  \node[right=0 of a-3] (x2) {};

  \path[linez] (a-1) -- node (xx2) {} node[cap2,yshift=-4pt] (s2) {$\Struct$}  (a-2);

  \path[lined] (a-0) -- (a-1);
  \path[lined] (a-2) -- (a-3);
  \node[lbl,below=3pt of a-1] (c1) {};
  \node[lbl,below=3pt of a-2] (c2){};

  \draw[bp,looseness=1]  (a-0) to (a-3);
  \draw[bp,out=80,in=100,looseness=.9]  (a-1) to (a-2);
  \node[xshift=-2pt] at (a-0.west) (y3) {};

\begin{pgfonlayer}{background}
  \node[rectangle,inner sep=2pt,draw,fit=(a-1.west)(a-2.east) (c1) (c2)] (r1) {};
  \path[block]   (r1.south west) to   (r1.south east) to (r1.north east) to[out=90,in=90,looseness=1.1] (r1.north west) to (r1.south west) ;
  \end{pgfonlayer}{background}
\end{scope}
\node[above= \LabSepB of a-1, caption] {{\bf Case 2:} Extremities are paired, surrounded by another base-pair, forming a stacking base-pair.};

\begin{scope}[xshift=\HSep,yshift=00pt]
  \node[base] (a-1) at (0,0) {\color{red}$b'$};
  \node[base] (a-2) at (5,0) {\color{red}$c'$};

  \node[basesmall,right=\BSep of a-1] (a-1b) {};
  \node[basesmall,left=\BSep of a-2] (a-2b)  {};

  \node[left=\BSep of a-1, basephantom] (a-0) {$a$};
  \node[right=\BSep of a-2, basephantom] (a-3) {$d$};

  \path[linez] (a-1b) -- node (xy2) {}  node[cap2,yshift=-4pt]  {$\Struct'$}  (a-2b);

  \path[line] (a-1) -- (a-1b);
  \path[line] (a-2) -- (a-2b);
  \path[lined] (a-0) -- (a-1);
  \path[lined] (a-2) -- (a-3);
  \node[lbl,below=3pt of a-1b] (c1) {};
  \node[lbl,below=3pt of a-2b] (c2){};

  \draw[bp,out=80,in=100,looseness=.9]  (a-1) to (a-2);
  \draw[bp,out=80,in=100,looseness=.9]  (a-0) to (a-3);
  \node[xshift=-2pt] at (a-0.west) (y2) {};

\begin{pgfonlayer}{background}
  \node[rectangle,inner sep=2pt,draw,fit=(a-1b.west)(a-2b.east) (c1) (c2)] (r1) {};
  \path[block]   (r1.south west) to  (r1.south east) to (r1.north east) to[out=90,in=90,looseness=1] (r1.north west) to (r1.south west) ;
  \end{pgfonlayer}{background}
\end{scope}
 
  \path (x2) edge[arr] (y2);
%%% Junction case %%%%
\begin{scope}[yshift=\VSepDown] 
  \node[base] (a-1) at (0,0) {$b$};
  \node[basesmall] (a-2) at (4,0) {};
  \node[base] (a-2k) at (2,0) {$c$};

  \node[left=\BSep of a-1, basephantom] (a-0) {$a$};
  \node[right=\BSep of a-2, basephantom] (a-3) {$d$};
  \node[right=0 of a-3] (x3) {};


  \path[linez] (a-1) --  (a-2k);
  \path[linez] (a-2k) --  (a-2);
  \path[draw=none] (a-1) -- node[cap2,yshift=6pt](s3) {$\Struct$}  (a-2);

  \path (a-1) --  node (xx3) {} (a-2);


  \path[lined] (a-0) -- (a-1);
  \path[lined] (a-2) -- (a-3);
  \node[lbl,below=3pt of a-1] (c1) {};
  \node[lbl,below=3pt of a-2] (c2){};
  \node[lbl,below=3pt of a-2k] (c3){};

  \draw[bp,looseness=.9,dashed]  (a-0) to (a-3);
  \draw[bp,out=80,in=100,looseness=.9]  (a-1) to (a-2k);

\begin{pgfonlayer}{background}
  \node[rectangle,inner sep=2pt,draw,fit=(a-1.west)(a-2.east) (c1) (c2)] (r1) {};
  \path[block]   (r1.south west) to (r1.south east) to (r1.north east) to[out=90,in=90,looseness=1.1] (r1.north west) to (r1.south west) ;
  \end{pgfonlayer}{background}
\end{scope}

\node[above= \LabSepC of a-1, caption] {{\bf Case 3:} First position in paired to some position, but not involved in a stacking pair.};

\begin{scope}[xshift=\HSep,yshift=\VSepDown]
  \node[base] (a-1) at (0,0) {\color{red}$b'$};
  \node[base] (a-p) at (3,0) {\color{red}$c'$};

  \node[basesmall,right=\BSep of a-1] (a-1b) {};
  \node[basesmall,left=\BSep of a-p] (a-pb)  {};
  \node[basesmall,right=\BSep of a-p] (a-pa)  {};

  \node[basesmall] (a-2) at (5,0) {};
  \node[left=\BSep of a-1, basephantom] (a-0) {};
  \node[right=\BSep of a-2, basephantom] (a-3) {$d$};

  \path[linez] (a-1b) -- node (xy3) {} node[cap2,yshift=-7pt] {$\Struct'$} (a-pb);
  \path[linez] (a-pa) -- node (xz3) {} node[cap2,yshift=-7pt] {$\Struct''$} (a-2);

  \path[line] (a-1) -- (a-1b);
  \path[line] (a-pb) -- (a-p);
  \path[line] (a-pa) -- (a-p);

  \path[lined] (a-0) -- (a-1);
  \path[lined] (a-2) -- (a-3);
  \node[lbl,below=3pt of a-1b] (c1) {};
  \node[lbl,below=3pt of a-2] (c5){};
  \node[lbl,below=3pt of a-p] (c3) {};
  \node[lbl,below=3pt of a-pb] (c2){};
  \node[lbl,below=3pt of a-pa] (c4){};

  \draw[bp]  (a-1) to[looseness=1.4] (a-p);
  \node[xshift=-2pt] at (a-0.west) (y3) {};

\begin{pgfonlayer}{background}

  \node[rectangle,inner sep=2pt,draw,fit=(a-1b.west)(a-pb.east) (c1) (c2)] (r1) {};
  \path[block]   (r1.south west) to (r1.south east) to (r1.north east) to[out=90,in=90,looseness=1.7] (r1.north west) to (r1.south west) ; 

  \node[rectangle,inner sep=2pt,draw,fit=(a-pa.west)(a-2.east) (c4) (c5)] (r2) {};
  \path[block]   (r2.south west) to  (r2.south east) to (r2.north east) to[out=90,in=90,looseness=1.7] (r2.north west) to (r2.south west) ;

  \end{pgfonlayer}{background}

\end{scope}



  \path (x3) edge[arr] (y3);

 \newcommand{\CapSep}{35pt}
 \tikzstyle{cap3}=[anchor=base,font=\relsize{+3}] 

 \node[below=\CapSep of xx1.center,cap3] {$m$};
 \node[ below=\CapSep of xx2.center,cap3] {$m$};
 \node[ below=\CapSep of xx3.center,cap3] {$m$};

 \node[below=\CapSep of xy1.center,cap3] {$m-{\color{red}\Kron_{b,b'}}$};
 \node[below=\CapSep of xy2.center,cap3] {$m-{\color{red}\Kron_{bc,b'c'}}$};
 \node[below=\CapSep of xy3.center,cap3] {$m'$};
 \node[below=\CapSep of xz3.center,text width=7em,cap3] {\;\,$m-m'-{\color{red}\Kron_{bc,b'c'}}$

};

}

  
\end{tikzpicture}2}
\caption{Principle of the inside computation (aka partition function). Any sequence with $m$ mutations over an interval $[i,j]$ 
can be decomposed as a sequence over $[i+1,j]$ preceded by a, possibly mutated, base at $i$ 
(Unpaired case), a sequence over $[i+1,j-1]$ surrounded by some base-pair (Stacking-pair case), 
or as two sequences over $[i+1,k-1]$ and $[k+1,j]$, completed by some base-pair (General base-pairing case). 
In each case, one has to investigate any possible ways to distribute mutations between the different sequences 
and locally instanciated bases.}
\end{figure}

The \emph{Inside} function $\Z{i,j}{m}{a,b}$ is the partition function, i.e. the sum of Boltzmann factors, over all sequences in the interval $[i,j]$, at distance $m$ of $s_{[i,j]}$, and having flanking nucleotides $a$ and $b$ (at positions $i-1$ and $j+1$ respectively). 
Such terms can be defined by recurrence, for which the following initial conditions holds:

\begin{equation}
	\forall i \in [0,n-1]:\, \Z{i+1,i}{m}{a,b}=\left\{
	\begin{array}{ll}
		1 &\text{If } m = 0\\
		0 &\text{Otherwise.}
	\end{array}\right.
\label{eq:Z_in}
\end{equation}
In other words, the set of sequences at distance $m$ of the empty sequence is either empty if $m>0$, or restricted to the empty sequence, having energy $0$, if $m=0$. Since the energetic terms only depend on base pairs, they are not involved in the initial conditions. 

The main recursion is composed of four terms:
\begin{equation}
	\Z{i,j}{m}{a,b}:=\left\{
  \begin{array}{ll}
  		\displaystyle
      \sum_{\substack{a'\in \B,\\ \Kron_{a',s_i}\le m}}  
      \Z{i+1,j}{m-\Kron_{a',s_i}}{a',b} & \text{If }S_{i}=-1\\
      \displaystyle
      \sum_{\substack{a',b'\in \B^2,\\ \Kron_{a'b',s_is_j}\le m}}
			 e^{\frac{-E_{(i,j),ab \to a'b'}^{\Omega,\beta}}{RT}}
			 \cdot \Z{i+1,j-1}{m-\Kron_{a'b',s_is_j}}{a',b'}&
			 \text{Elif }S_i=j \land S_{i-1}=j+1\\
			 \displaystyle
      \sum_{\substack{a',b'\in \B^2,\\ \Kron_{a'b',s_is_k}\le m}}
      \sum_{m'=0}^{m-\Kron_{a'b',s_is_k}}
   		 e^{\frac{-E_{(i,k),\varnothing\to a'b'}^{\Omega,\beta}}{RT}}
      \cdot\Z{i+1,k-1}{m-\Kron_{a'b',s_is_k}-m'}{a',b'}
      \cdot\Z{k+1,j}{m'}{b',b} & \text{Elif }S_i=k \land i < k \leq j\\
      0 &\text{Otherwise}
	\end{array}\right.
\label{eq:Z_rec}
\end{equation}
The cases can be broken down as follows:
\begin{description}
\item[$S_{i}=-1$:] If the nucleotide at position $i$ is unpaired, then 
any sequence consists in a, possibly mutated, nucleotide $a'$ at position  $i$, 
followed by a sequence over $[i+1,j]$ having either $m-\Kron_{a',s_i}$, accounting for a possible mutation 
at position $i$, and having flanking nucleotides $a'$ and $b$.
\item[$S_i=j$ and $S_{i-1}=j+1$:] Any sequence generated in $[i,j]$ consists of two, possibly mutated, nucleotides $a'$ and $b'$, flanking a sequence over $[i+1,j-1]$ having distance $m-\Kron_{a'b',s_is_j}$ (to avoid exceeding the targeted distance $m$).
Since positions $i$ and $i-1$ are paired with $j$ and $j+1$ respectively, 
then a stacking energy contribution is added. 
\item[$S_i=k$ and $i<k \leq j$:] If position $i$ is paired and not involved in a stacking, then the 
only term contributing directly to the energy is the isostericity of the base pair $(i,k)$. 
Any sequence on $[i,j]$ consists of two nucleotide $a'$ and $b'$ at positions $i$ and $k$ respectively, flanking a sequence over interval $[i+1,k-1]$ and preceding a (possibly empty) sequence interval $[k+1,j]$. Since the number of mutations sum to $m$ over the whole sequence must , then a parameter $m'$ is introduced to distribute the remaining mutations between the two sequences.
\item[Else:] In any other case, we are in a derivation of the SCFG that does not correspond to the secondary structure $S$, and we return $0$.
\end{description}

\subsubsection{Outside}	
\begin{figure}
\resizebox{\textwidth}{!}{%!TEX root = main_RECOMB.tex

\begin{tikzpicture}
  \definecolor{rougeForb}{HTML}{eb23238f}
  \definecolor{rougeForbP}{HTML}{6d1515ff}

  \newcommand{\BSep}{9pt}
  \newcommand{\HSep}{350pt}
  \newcommand{\RelPosA}{0pt}
  \newcommand{\RelPosB}{-130pt}
  \newcommand{\RelPosC}{-260pt}
  \newcommand{\RelPosD}{-390pt}
  \newcommand{\FitSep}{3.5pt}

  \newcommand{\LabSepB}{15pt}


  \newcommand{\CaptionTxtA}{{\bf Case 1}: Next leftward position is unpaired.}
  \newcommand{\CaptionTxtB}{{\bf Case 2}: Paired extremal positions, nesting another base-pair, forming a stacking base-pair.}
  \newcommand{\CaptionTxtC}{{\bf Case 3}: Next leftward position is paired to the right, but no stacking pairs.}
  \newcommand{\CaptionTxtD}{{\bf Case 4}: Next leftward position is paired to the left.}


  \tikzstyle{caption}=[%fill=gray!20,draw=gray!60,thick,inner sep=4pt,rounded corners=6pt,
font=\relsize{+3}\sffamily,anchor=north west,xshift=-40pt]


  \tikzstyle{basebase}=[circle,draw,thick,inner sep=0,minimum width=18pt,fill=white,font=\relsize{+2}]

  \tikzstyle{base}=[basebase]
  \tikzstyle{basesmall}=[basebase,minimum width=10pt]
  \tikzstyle{basephantom}=[basebase,dashed]
  \tikzstyle{linez}=[draw,snake=zigzag, segment aspect=.2,%
line after snake=0pt,  
        segment length=10pt,thick]
  \tikzstyle{lined}=[linez,draw,snake=none,thick]
  \tikzstyle{line}=[linez,draw,snake=none,thick]
  \tikzstyle{lineh}=[linez]
  \tikzstyle{bp}=[in=90,out=90,draw,line width=1.5pt,blue,looseness=1.7]
  \tikzstyle{blockin}=[trapezium,trapezium angle=83,  fill=blue!20, draw=blue!20!gray,line width=1.5pt, inner sep=0,drop shadow]
  \tikzstyle{blockout}=[blockin,draw=red!80!white!55!gray,fill=red!40!white!95!gray,line width=1.5pt, drop shadow]
  \tikzstyle{lbl}=[inner sep=0,font=\relsize{+3}]
  \tikzstyle{arr}=[-open triangle 60,line width=1.5pt]


 %%%%%%% Unpaired %%%%%%%
  \begin{scope}[yshift=\RelPosA]
 %%%%%%% LHS %%%%%%%
  \begin{scope}[xshift=-\HSep]
  \node[basesmall] (n-beg) at (0,0) {};
  \node[basesmall] (a-0b) at (1.4,0) {};
  \node[base] (a-0) at (2,0) {$a$};
  \node[basesmall] (a-3) at (6,0) {};
  \node[basesmall] (n-end) at (8,0) {};
  \node[right=\BSep of a-0, basephantom] (a-1) {$b$};
  \node[left=\BSep of a-3, basephantom] (a-2) {$c$};
  \path[lineh] (a-1) --  node[lbl,pos=.5,above=45pt] (lbl1) {$m$}  (a-2);
  \path[lined] (a-0) -- (a-1);
  \path[lined] (a-2) -- (a-3);
  \path[linez] (n-beg) -- (a-0b);
  \path[lined] (a-0b) -- (a-0);
  \path[linez] (a-3) -- (n-end);

  \node[right=5pt of n-end] (x) {};



  \begin{pgfonlayer}{background}
  \node[rectangle,inner sep=\FitSep,draw,fit=(n-beg)(a-0.east)] (r1) {};
  \node[rectangle,inner sep=\FitSep,draw,fit=(n-end)(a-3.west)] (r2) {};
  \path[blockout]   (r1.south west) to (r1.south east) to (r1.north east) to[out=90,in=90,looseness=0.8] (r2.north west) to (r2.south west) to (r2.south east) to (r2.north east) to[out=90,in=90,looseness=0.9] (r1.north west) to (r1.south west) ;
  \end{pgfonlayer}{background}

\node[below= \LabSepB of n-beg, caption] {\CaptionTxtA};

  \end{scope}
 %%%%%%% /LHS %%%%%%%



  \node[basesmall] (n-beg) at (0,0) {};
  \node[base, drop shadow] (a-0) at (2,0) {\color{StressColor}$a'$};
  \node[basesmall,left=\BSep of a-0] (a-0b) {};
  \node[basesmall] (a-3) at (6,0) {};
  \node[basesmall] (n-end) at (8,0) {};
  \node[right=\BSep of a-0, basephantom] (a-1) {$b$};
  \node[left=\BSep of a-3, basephantom] (a-2) {$c$};
  \path[lineh] (a-1) --  node[lbl,pos=.5,above=40pt] (lbl1) {$m-{\color{StressColor}\delta_{a,a'}}$}  (a-2);
  \path[lined] (a-0) -- (a-1);
  \path[lined] (a-2) -- (a-3);
  \path[lined] (a-0) -- (a-0b);
  \path[linez] (n-beg) -- (a-0b);
  \path[linez] (a-3) -- (n-end);



  \node[left=5pt of n-beg] (y1) {};

  \path (x) edge[arr]    (y1);

  \begin{pgfonlayer}{background}
  \node[rectangle,inner sep=\FitSep,draw,fit=(n-beg)(a-0b.east)] (r1) {};
  \node[rectangle,inner sep=\FitSep,draw,fit=(n-end)(a-3.west)] (r2) {};
  \path[blockout]   (r1.south west) to (r1.south east) to (r1.north east) to[out=90,in=90,looseness=0.8] (r2.north west) to (r2.south west) to (r2.south east) to (r2.north east) to[out=90,in=90,looseness=0.9] (r1.north west) to (r1.south west) ;
  \end{pgfonlayer}{background}
  \end{scope}



 %%%%%%% /Unpaired %%%%%%%


 %%%%%%% Stacking %%%%%%%
  \begin{scope}[yshift=\RelPosB]

 %%%%%%% LHS %%%%%%%
  \begin{scope}[xshift=-\HSep]
  \node[basesmall] (n-beg) at (0,0) {};
  \node[basesmall] (a-0b) at (1.4,0) {};
  \node[base] (a-0) at (2,0) {$a$};
  \node[base] (a-3) at (6,0) {$d$};
  \node[basesmall] (a-3b) at (6.6,0) {};
  \node[basesmall] (n-end) at (8,0) {};
  \node[right=\BSep of a-0, basephantom] (a-1) {$b$};
  \node[left=\BSep of a-3, basephantom] (a-2) {$c$};
  \path[lineh] (a-1) --  node[lbl,pos=.5,above=45pt] (lbl1) {$m$}  (a-2);
  \path[lined] (a-0) -- (a-1);
  \path[lined] (a-2) -- (a-3);
  \path[linez] (n-beg) -- (a-0b);
  \path[lined] (a-0b) -- (a-0);
  \path[lined] (a-3) -- (a-3b);
  \path[linez] (a-3b) -- (n-end);

  \node[right=5pt of n-end] (x) {};

  \draw[bp,looseness=.9] (a-0) to (a-3);
  \draw[bp,looseness=.9] (a-1) to (a-2);


  \begin{pgfonlayer}{background}
  \node[rectangle,inner sep=\FitSep,draw,fit=(n-beg)(a-0.east)] (r1) {};
  \node[rectangle,inner sep=\FitSep,draw,fit=(n-end)(a-3.west)] (r2) {};
  \path[blockout]   (r1.south west) to (r1.south east) to (r1.north east) to[out=90,in=90,looseness=0.8] (r2.north west) to (r2.south west) to (r2.south east) to (r2.north east) to[out=90,in=90,looseness=0.9] (r1.north west) to (r1.south west) ;
  \end{pgfonlayer}{background}
  \end{scope}

\node[below= \LabSepB of n-beg, caption] {\CaptionTxtB};

 %%%%%%% /LHS %%%%%%%


  \node[basesmall] (n-beg) at (0,0) {};
  \node[base, drop shadow] (a-0) at (2,0) {\color{StressColor}$a'$};
  \node[basesmall,left=\BSep of a-0] (a-0b) {};
  \node[base, drop shadow] (a-3) at (6,0) {\color{StressColor}$d'$};
  \node[basesmall,right=\BSep of a-3] (a-3b) {};
  \node[basesmall] (n-end) at (8,0) {};
  \node[right=\BSep of a-0, basephantom] (a-1) {$b$};
  \node[left=\BSep of a-3, basephantom] (a-2) {$c$};
  \path[lineh] (a-1) --  node[lbl,pos=.5,above=45pt] (lbl1) {$m-{\color{StressColor}\delta_{ad,a'd'}}$}  (a-2);
  \path[lined] (a-0) -- (a-1);
  \path[lined] (a-2) -- (a-3);
  \path[lined] (a-0) -- (a-0b);
  \path[lined] (a-3b) -- (a-3);
  \path[linez] (n-beg) -- (a-0b);
  \path[linez] (a-3b) -- (n-end);
  \draw[bp,looseness=.8] (a-1) to (a-2);
  \draw[bp,looseness=.8] (a-0) to (a-3);




  \node[left=5pt of n-beg] (y2) {};

  \path (x) edge[arr]    (y2);


  \begin{pgfonlayer}{background}
  \node[rectangle,inner sep=\FitSep,draw,fit=(n-beg)(a-0b.east)] (r1) {};
  \node[rectangle,inner sep=\FitSep,draw,fit=(n-end)(a-3b.west) ] (r2) {};
  \path[blockout]   (r1.south west) to (r1.south east) to (r1.north east) to[out=90,in=90,looseness=0.8] (r2.north west) to (r2.south west) to (r2.south east) to (r2.north east) to[out=90,in=90,looseness=0.9] (r1.north west) to (r1.south west) ;
  \end{pgfonlayer}{background}
  \end{scope}
 %%%%%%% /Stacking %%%%%%%

 %%%%%%% BPRight %%%%%%%


  \begin{scope}[yshift=\RelPosC]

 %%%%%%% LHS %%%%%%%
  \begin{scope}[xshift=-\HSep]
  \node[basesmall] (n-beg) at (0,0) {};
  \node[base] (a-0) at (1.3,0) {$a$};
  \node[basesmall] (a-3) at (4.5,0) {};
  \node[basesmall] (a-4b) at (5.8,0) {};
  \node[base] (a-4) at (6.4,0) {$d$};
  \node[basesmall] (a-4t) at (7,0) {};
  \node[basesmall] (n-end) at (8,0) {};
  \node[right=\BSep of a-0, basephantom] (a-1) {$b$};
  \node[left=\BSep of a-3, basephantom] (a-2) {$c$};
  \path (n-beg) --  node[lbl,pos=.5,above=45pt] (lbl1) {$m$}  (n-end);
  \path[lineh] (a-1) --   (a-2);
  \path[lined] (a-0) -- (a-1);
  \path[lined] (a-2) -- (a-3);
  \path[linez] (n-beg) -- (a-0);
  \path[linez] (a-3) -- (a-4b);
  \path[line] (a-4b) -- (a-4);
  \path[line] (a-4) -- (a-4t);
  \path[linez] (a-4t) -- (n-end);

  \draw[bp,looseness=.7] (a-0) to (a-4);
  \draw[bp,looseness=.9,dashed] (a-1) to (a-2);


  \node[right=5pt of n-end] (x) {};

  \begin{pgfonlayer}{background}
  \node[rectangle,inner sep=\FitSep,draw,fit=(n-beg)(a-0.east)] (r1) {};
  \node[rectangle,inner sep=\FitSep,draw,fit=(n-end)(a-3.west) ] (r2) {};
  \path[blockout]   (r1.south west) to (r1.south east) to (r1.north east) to[out=90,in=90,looseness=0.8] (r2.north west) to (r2.south west) to (r2.south east) to (r2.north east) to[out=90,in=90,looseness=0.9] (r1.north west) to (r1.south west) ;
  \end{pgfonlayer}{background}
  \end{scope}

\node[below= \LabSepB of n-beg, caption] {\CaptionTxtC};

 %%%%%%% /LHS %%%%%%%

  \node[basesmall] (n-beg) at (0,0) {};
  \node[base,drop shadow] (a-0) at (1.8,0) {\color{StressColor}$a'$};
  \node[basesmall,left=\BSep of a-0] (a-0b) {};
  \node[base,drop shadow] (a-3) at (6.25,0) {\color{StressColor}$d'$};
  \node[basesmall] (b-1) at (4.0,0) {};
  \node[basesmall,left=\BSep of a-3] (b-2){};
  \node[basesmall,right=\BSep of a-3] (a-3b) {};
  \node[basesmall] (n-end) at (8,0) {};
  %\node[right=\BSep of a-0, basephantom] (a-1) {$a$};
  \node[left=\BSep of b-1, basephantom] (a-2) {$c$};
  \path[lineh] (a-0) --   (a-2);
  \path[lined] (a-2) -- (b-1);
  \path[linez] (b-1) -- node[lbl,pos=.5,above=8pt] (lbl1) {$m'$}  (b-2);
  \path (n-beg) -- node[lbl,pos=.5,above=46pt] (lbl1) {$m-m'-{\color{StressColor}\delta_{ad,a'd'}}$}  (n-end);
  \path[lined] (b-2) -- (a-3);
  \path[lined] (a-0) -- (a-0b);
  \path[lined] (a-3b) -- (a-3);
  \path[linez] (n-beg) -- (a-0b);
  \path[linez] (a-3b) -- (n-end);
  \draw[bp,looseness=.8] (a-0) to (a-3);

  \node[left=5pt of n-beg] (y3) {};
  \path (x) edge[arr]    (y3);

  \begin{pgfonlayer}{background}
  \node[rectangle,inner sep=\FitSep,draw,fit=(n-beg)(a-0b.east)] (r1) {};
  \node[rectangle,inner sep=\FitSep,draw,fit=(n-end)(a-3b.west) ] (r2) {};
  \path[blockout]   (r1.south west) to (r1.south east) to (r1.north east) to[out=90,in=90,looseness=0.8] (r2.north west) to (r2.south west) to (r2.south east) to (r2.north east) to[out=90,in=90,looseness=0.9] (r1.north west) to (r1.south west) ;

  \node[rectangle,inner sep=\FitSep,draw,fit=(b-1)(b-2) ] (r3) {};
  \path[blockin]   (r3.south west) to (r3.south east) to (r3.north east) to[out=90,in=90,looseness=1.4] (r3.north west) to (r3.south west) ;
  \end{pgfonlayer}{background}
  \end{scope}
 %%%%%%% /BPRight %%%%%%%


 %%%%%%% BP Left %%%%%%%
  \begin{scope}[yshift=\RelPosD]

 %%%%%%% LHS %%%%%%%
  \begin{scope}[xshift=-\HSep]
  \node[basesmall] (n-beg) at (0,0) {};
  \node[basesmall] (a-w) at (1.1,0) {};
  \node[base] (a-x) at (1.7,0) {$a$};
  \node[basesmall] (a-y) at (2.3,0) {};
  \node[basesmall] (a-z) at (3.3,0) {};
  \node[base] (a-0) at (3.9,0) {$b$};
  \node[basesmall] (a-3) at (7,0) {};
  \node[basesmall] (n-end) at (8,0) {};
  \node[right=\BSep of a-0, basephantom] (a-1) {$c$};
  \node[left=\BSep of a-3, basephantom] (a-2) {$d$};
  \path[lineh] (a-1) --  (a-2);
  \path (n-beg) --  node[lbl,pos=.5,above=45pt] (lbl1) {$m$}  (n-end);
  \path[lined] (a-0) -- (a-1);
  \path[lined] (a-2) -- (a-3);
  \path[linez] (n-beg) -- (a-w);
  \path[line] (a-w) -- (a-x);
  \path[linez] (a-x) -- (a-y);
  \path[linez] (a-y) -- (a-z);
  \path[line] (a-z) -- (a-0);
  \path[linez] (a-3) -- (n-end);

  \node[right=5pt of n-end] (x) {};

  \draw[bp,looseness=1.05] (a-x) to (a-0);
  \draw[bp,looseness=.9,dashed] (a-1) to (a-2);


  \begin{pgfonlayer}{background}
  \node[rectangle,inner sep=\FitSep,draw,fit=(n-beg)(a-0.east)] (r1) {};
  \node[rectangle,inner sep=\FitSep,draw,fit=(n-end)(a-3.west) ] (r2) {};
  \path[blockout]   (r1.south west) to (r1.south east) to (r1.north east) to[out=90,in=90,looseness=0.8] (r2.north west) to (r2.south west) to (r2.south east) to (r2.north east) to[out=90,in=90,looseness=0.9] (r1.north west) to (r1.south west) ;
  \end{pgfonlayer}{background}
  \end{scope}
 
\node[below= \LabSepB of n-beg, caption] {\CaptionTxtD};


%%%%%%% /LHS %%%%%%%

  \node[basesmall] (n-beg) at (0,0) {};
  \node[base, drop shadow] (a-0) at (1.8,0) {\color{StressColor}$a'$};
  \node[basesmall,left=\BSep of a-0] (a-0b) {};
  \node[base, drop shadow] (a-3) at (4.5,0) {\color{StressColor}$b'$};
  \node[basesmall,right=\BSep of a-0] (b-1)  {};
  \node[basesmall,left=\BSep of a-3] (b-2){};
  \node[basesmall] (a-3b)  at (6.75,0) {};
  \node[basephantom,left=\BSep of a-3b] (a-3c){$d$};
  \node[basesmall] (n-end) at (8,0) {};


  %\node[right=\BSep of a-0, basephantom] (a-1) {$a$};
  %\node[left=\BSep of b-1, basephantom] (a-2) {$b$};
  \path[lined] (b-1) -- (a-0);
  \path[linez] (b-1) -- node[lbl,pos=.5,above=8pt] (lbl1) {$m'$}  (b-2);
  \path (n-beg) -- node[lbl,pos=.5,above=43pt] (lbl1) {$m-m'-{\color{StressColor} \delta_{ab,a'b'}}$}  (n-end);
  \path[lined] (a-3b) -- (a-3c);
  \path[lined] (a-0) -- (a-0b);
  \path[lined] (b-2) -- (a-3);
  \path[lineh] (a-3c) -- (a-3);
  \path[linez] (n-beg) -- (a-0b);
  \path[linez] (a-3b) -- (n-end);
  \draw[bp,looseness=1.05] (a-0) to (a-3);

  \node[left=5pt of n-beg] (y4) {};
  \path (x) edge[arr]    (y4);


  \begin{pgfonlayer}{background}
  \node[rectangle,inner sep=\FitSep,draw,fit=(n-beg)(a-0b.east)] (r1) {};
  \node[rectangle,inner sep=\FitSep,draw,fit=(n-end)(a-3b.west) ] (r2) {};
  \path[blockout]   (r1.south west) to (r1.south east) to (r1.north east) to[out=90,in=90,looseness=0.8] (r2.north west) to (r2.south west) to (r2.south east) to (r2.north east) to[out=90,in=90,looseness=0.9] (r1.north west) to (r1.south west) ;

  \node[rectangle,inner sep=\FitSep,draw,fit=(b-1)(b-2)] (r3) {};
  \path[blockin]   (r3.south west) to (r3.south east) to (r3.north east) to[out=90,in=90,looseness=1.4] (r3.north west) to (r3.south west) ;
  \end{pgfonlayer}{background}
  \end{scope}
 %%%%%%% /BP Left %%%%%%%

\end{tikzpicture}}
\caption{Principle of the outside computation. Note that the outside algorithm uses intermediate results from the inside algorithm, 
therefore its efficient implementation requires an implementation of the inside computation.}
\end{figure}
The \emph{Outside} function, $\mathcal Y$, is the partition function considering only the 
contributions of subsequences $[0,i]\cup[j,n-1]$ over the mutants of $s$ having exactly $m$ mutations between $[0,i]\cup[j,n-1]$ and whose nucleotide at position $i+1$ is $a$ 
(resp. in position $j-1$ it is $b$).
We define function $\Y{i,j}{m}{a,b}$ as a recurrence, and will use as initial conditions:
\begin{equation}
	\Y{-1,j}{m}{X,X}:=
		\displaystyle
	  \Z{j,n-1}{m}{X,X}
\label{eq:Y_in}
\end{equation}
The recurrence, as shown below, will increase the interval $[i,j]$ by decreasing $i$ when
it is not base paired. If it is with a position $k>j$, we increase $j$ to include it.
 Thus, when we need
to evaluate an interval as $(-1,j)$, all stems between $(0,j)$ are taken into account and the
structure between $(j,n-1)$ must be a set of independent stems. Therefore,
 all the outside energy between $[j,n-1]$ is
equal to $\Z{j,n-1}{m}{X,X}$, for any $X\in B$. The recursion itself is as follows.
\begin{equation}
	\Y{i,j}{m}{a,b} = \left\{
  \begin{array}{ll}
		\displaystyle
    \sum_{\substack{a'\in \B,\\ \Kron_{a',s_i}\le m}}
    \Y{i-1,j}{m- \Kron_{a',s_i}}{a',b} &
    \text{Elif }S_i=-1 \\
    \displaystyle
    \sum_{\substack{a'b'\in \B^2,\\ \Kron_{a'b',s_is_j}\le m}}
		 e^{\frac{-E_{(i,j),ab \to a'b'}^{\Omega,\beta}}{RT}}\cdot
    \Y{i-1,j+1}{m- \Kron_{a'b',s_is_j}}{a',b'} &
   	 \text{Elif }S_{i}=j \land S_{i+1}=j-1\\
		 \displaystyle
		 \sum_{\substack{a'b'\in \B^2,\\ \Kron_{a'b',s_is_k}\le m}}
		 \sum_{m'=0}^{m-\Kron_{a'b',s_is_k}}
  		 e^{\frac{-E_{(i,k),\varnothing\to a'b'}^{\Omega,\beta}}{RT}}
		 \cdot\Y{i-1,k+1}{m- \Kron_{a'b',s_is_k} - m'}{a',b'}
     \cdot\Z{j,k-1}{m'}{b,b'} &
		 \text{Elif }S_{i}=k \geq j\\
		 \displaystyle
		 \sum_{\substack{a'b'\in \B^2,\\ \Kron_{a'b',s_ks_i}\le m}}
		 \sum_{m'=0}^{m-\Kron_{a'b',s_ks_i}}
   	 e^{\frac{-E_{(k,i),\varnothing\to a'b'}^{\Omega,\beta}}{RT}}
		 \cdot\Y{k-1,j}{m- \Kron_{a'b',s_ks_i} - m'}{a',b}
     \cdot\Z{k+1,i-1}{m'}{a',b'} &
		 \text{Elif }-1 < S_{i}=k < i\\
		 0 & \text{Otherwise}
  \end{array}\right.
\label{eq:Y_rec}
\end{equation}
The five cases can be broked down as follows.
\begin{description}
\item[$S_i=-1$:] If the nucleotide at position $i$ is not paired, then the value is the same
as if we decrease the lower interval bound by $1$ (i.e. $i-1$), and consider all possible
nucleotides $a'$ at position $i$, correcting the number of mutants
in function of $\Kron_{a',s_i}$.
\item[$S_{i}=j$ and $S_{i+1}=j-1$:] If nucleotide $i$ is paired with $j$ and nucleotide $i+1$ is
paired with $j-11$, we are in the only case were stacked base pairs can occur. We thus add
the energy of the stacking and of the isostericity of the base pair $(i,j)$. What is left
to compute is the \emph{outside} value for the interval $[i-1,j+1]$ over all possible nucleotides 
$a',b'\in B^2$ at positions $i$ and $j$ respectively.
\item[$S_{i}=k \geq j$:]If nucleotide $i$ is paired with position $k\geq j$, 
and is not stacked inside, the 
only term contributing directly to the energy is the isostericity of the base pair $(i,k)$. 
We can then consider the outside interval $[i-1,k+1]$ by multiplying it by the the \emph{inside}
value of the newly included interval (i.e. $[j,k-1]$), for 
all possible values $a',b'\in B^2$ for nucleotides at positions $i$ and $k$ respectively.
\item[$-1<S_{i}<i$:]As above but if position $i$ is to pairing with a lower value.
\item[Else:] In all other cases, we are in a derivation of the SCFG that does not correspond to the 
secondary structure $S$, and we return $0$.


\end{description}

\subsubsection{Combining Inside and Outside values into point-wise mutations probabilities}
By construction, the partition function over all sequences at exactly $m$ mutations of $s$ can 
be described in function of the \emph{inside} term as $\Z{0,n-1}{m}{X,X}$,
 for any nucleotide $X\in B$ or
in function of the \emph{outside} term, for any unpaired position $k$:
$$
	\Z{0,n-1}{m}{X,X}
	\equiv
	\sum_{\substack{a\in \B,\\ \Kron_{a,s[k]}\le m}}	
	\Y{k-1,k+1}{m-\Kron_{a,s[k]}}{a,a}
$$

We are now left to compute the probability that a given position is a given nucleotide.
We leverage the \emph{Inside-Outside} construction to immediately obtain the following $3$ cases.
Given $i\in[0,n-1],x\in B$, and $M\geq 0$ a bound on the number of allowed mutations. 
\begin{align}
	\mathbb{P}(s_i = x\mid s,\Omega, S,M) &:= \frac{\mathcal{W}(i,x,s,\Omega,S,M)}{\sum_{m=0}^{M}\Z{0,n-1}{m}{X,X}}\label{eq:normalize}\\ 
\mathcal{W}(i,x,s,\Omega,S,M)&=
 \left\{
	\begin{array}{ll}
			\sum_{m=0}^{M}
			\Y{i-1,i+1}{m-\Kron_{x,s_i}}{x,x}
		&\text{If }S_i = -1\\
			\displaystyle
			\sum_{m=0}^{M}
			\sum_{\substack{b\in B\\\Kron_{xb,s_is_k\leq m}}}
			\sum_{m'=0}^{m-\Kron_{xb,s_is_k}}
     	 e^{\frac{-E_{(i,k),\varnothing\to xb}^{\Omega,\beta} }{RT}}
			\cdot\Y{i-1,k+1}{m-\Kron_{xb,s_is_k-m'}}{x,b}
			\cdot\Z{i+1,k-1}{m'}{x,b}
		&\text{If }S_i=k>i\\
    \displaystyle
			\sum_{m=0}^{M}
			\sum_{\substack{b\in B\\\Kron_{bx,s_ks_i\leq m}}}
			\sum_{m'=0}^{m-\Kron_{bx,s_ks_i}}
     	 e^{\frac{-E^{\Omega, \beta}_{(k,i),\varnothing\to bx}}{RT}}
			\cdot\Y{k-1,i+1}{m-\Kron_{bx,s_ks_i-m'}}{b,x}
			\cdot\Z{k+1,i-1}{m'}{b,x}
		&\text{If }S_i=k<i
	\end{array}\right.\label{eq:combine}
\end{align}

In every case, the denominator is the sum of the partitions function of exactly $m$ mutations, 
for $m$ smaller or equal to our target $M$. The numerators are divided in the following three cases.
\begin{description}
\item[$S_i=-1$:] If the nucleotide at position $i$ is not paired, we are concerned by the weights
over all sequences which have at position $i$ nucleotide $x$, which is exactly the sum
of the values of $\Y{i-1,i+1}{m-\Kron_{x,s_i}}{x,x}$, for all $m$ between $0$ and $M$.
\item[$S_i=k>i$:] Since we need to respect the derivation of the secondary structure $S$, if 
position $i$ is paired, we must consider the two partition functions. The \emph{outside} of the 
base pair, and the \emph{inside}, for all possible values for the nucleotide at position $k$, and
all possible distribution of the mutant positions between the inside and outside of the base pair. We also add the term of isostericity for this specific base pair.
\item[$S_i=k<i$:] Same as above, but with position $i$ pairing with a lower position.
\end{description}

\subsection{Complexity considerations}
Equations~\eqref{eq:Z_rec} and~\eqref{eq:Y_rec} can be computed using dynamic programming. Namely, the $\mathcal{Z}^{*}_{*}$ and $\mathcal{Y}^{*}_{*}$ terms are computed starting from smaller values of $m$ and interval lengths, memorizing the results as they become available to ensure constant-time access during later stages of the computation. Furthermore, energy terms $E(\cdot)$ can be accessed in constant time thanks to a simple precomputation (not described)  of the isostericity contributions in $\Theta(n\cdot|\Omega|)$. Computing any given term therefore requires $\Theta(m)$ operations.

In principle, $\Theta(m\cdot n^2)$ terms, identified by $(m,i,j)$ triplets, should be computed.
However, a close inspection of the recurrences reveals that the computation can be safely restricted to a subset of intervals $(i,j)$.
For instance, the inside algorithm only requires computing intervals $[i,j]$ that do not break any base-pair, and whose next position $j+1$ is either past the end of the sequence, or is base-paired prior to $i$. Similar constraints hold for the outside computation, resulting in a drastic limitation of the combinatorics of required computations, dropping from $\Theta(n^2)$ to $\Theta(n)$ the number of terms that need to be computed and stored. Consequently the overall complexity of the algorithm is $\Theta(n\cdot(|\Omega|+m^2))$ arithmetic operations and $\Theta(n\cdot(|\Omega|+m))$ memory.
